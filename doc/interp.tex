%\batchmode
\nonstopmode

%%% \documentclass{acmconf}     %% ICFP
\documentclass{llncs}[12pt]             %% Portugal

\usepackage[latin1]{inputenc}
\usepackage[T1]{fontenc}

\usepackage{proof}
%\pagestyle{empty}
%\thispagestyle{empty}

\newcommand\bs{\char '134}
%\newcommand\notyet[1]{#1}
\newcommand\notyet[1]{}
\newcommand\com[1]{}
\newcommand\nocom[1]{#1}
\newcommand\tmp[1]{{\sf #1}}

%\leftmargin -2cm
%\leftmargini 0cm
%\leftmarginv 0cm
%\leftmarginvi 0cm
%\textwidth 16cm
%\topmargin -1.0cm
%\textheight 24.5cm
%\setlength\leftmargin {-2cm}

\newcommand{\nt}[1]{{\it #1}}
\newcommand{\te}[1]{{\tt #1}}
\newcommand{\spp}{\mbox{~~}}

\newcommand{\ruleh}[3]{\infer [$\sf \scriptsize #1$] {#3} {#2}}
\newcommand{\seph}{ & }
\newcommand{\judge}[2]{#1 \vdash #2}
\newcommand{\judget}[2]{#1 & \vdash & #2}
\newcommand{\hastype}[2]{#1 \in #2}
%\newcommand{\rsep}{\vspace*{5mm}}
\newcommand{\rsep}{\medskip\medskip}
\newcommand{\dc}{$\te{::}$}
\newcommand{\dt}{$\te{.}$}
\newcommand{\sm}{$\te{;}$}
\newcommand{\teq}{$\te{=}$}
\newcommand{\fun}{\rightarrow}
\newcommand{\lam}{\lambda}
\newcommand{\eq}[2]{#1 \approx #2}
\newcommand{\hash}[1]{\;$\te{\#}$_{#1}}
\newcommand{\keyw}[1]{$\te{#1}$\;}

\newcommand{\subst}[3]{#1[#2 \mapsto #3]}

\def\mapstou{\mapstochar\rightharpoonup}
\newcommand{\eval}[2]{#1 & $\longmapsto$ & #2 \\}
\newcommand{\evalu}[2]{#1 & $\mapstou$ & #2 \\}
\newcommand{\ieval}[2]{$ #1 \longmapsto #2 $}
\newcommand{\ievalu}[2]{$ #1 \mapstou #2 $}
\newcommand{\imeval}[2]{$ #1 \longmapsto^* #2 $}
\newcommand{\imevalu}[2]{$ #1 \mapstou^* #2 $}


\author{
Lennart Augustsson \\
Magnus Carlsson \\
}
%%\affiliation{
\institute{
{ Department of Computing Sciences}\\
{ Chalmers University of Technology}\\
{ S-412 96 G{\"o}teborg, Sweden}\\
Email: {\tt \{augustss,magnus\}@cs.chalmers.se}\\
WWW: {\tt http://www.cs.chalmers.se/\~{}\{augustss,magnus\}}\\
}
\date{}
\title{
An exercise in dependent types: \\
A well-typed interpreter
}

\begin{document}
\maketitle

\begin{abstract}
The result type of an interpreter written in a typed language
is normally a tagged union.
By using depent types, we can be
more precise about the type of values that the intepreter
returns. There is no need for tagging these values with their
type, something which opens the door to more efficient interpreters. 
\end{abstract}

%% \copyrightspace %% ACM

\section{An annoying problem}

\tmp{We are ignoring $\perp$ for now.}

Writing an interpreter for a language which can produce results
of different types requires some kind of encoding of the result,
as in figure~\ref{fig:simple}.  The encoding is captured
in the tagging in the data type \te{Value}.  This tagging is inefficient
since each return value must be tagged. In addition, the tags
must be tested and the real values extracted again when they
are used.

\begin{figure*}
\begin{verbatim}
data Expr = EBool Bool | EInt Int | EAdd Expr Expr

data Value = VBool Bool | VInt Int | VError

interp :: Expr -> Value
interp (EBool b) = VBool b
interp (EInt i)  = VInt i
interp (EAdd e1 e2) =
    case (interp e1, interp e2) of
    (VInt i1, VInt i2) -> VInt (i1+i2)
    _                  -> VError
\end{verbatim}
\caption{A very simple language and its interpreter.}
\label{fig:simple}
\end{figure*}

The tags serve two purposes: first, they turn the result
into a union type so that the result is a {\em single} type;
second, they are used to make sure that the interpreted program
is type correct.  However, if the interpreted program has already
been type checked and it is known to be type correct so using the union
is really unnecessary.

Assuming a type correct program, the interpreter we would like to
write is shown in figure~\ref{fig:untyped}.  This interpreter can
be written in a dynamically typed langue like Lisp \cite{lisp}
or Erlang \cite{armstrong:erlang}, but
unfortunately not in strongly typed languages like Haskell \cite{haskell12} or
SML \cite{milner:sml-standard}.

\begin{figure}
\begin{verbatim}
interp (EBool b) = b
interp (EInt i)  = i
interp (EAdd e1 e2) = interp e1 + interp e2
\end{verbatim}
\caption{An untyped interpreter.}
\label{fig:untyped}
\end{figure}

\section{Dependent types to the rescue}
The problem with the untagged interpreter (figure~\ref{fig:untyped})
is that the type of the result {\em depends} on the value of the
argument.  To handle that, we need a more powerful type system,
one that can handle {\em dependent types}.  We are going to continue
this example by using Cayenne, \cite{augustsson:cayenne}, a Haskell-like
language which has dependent types.

The interpreter written in Cayenne is shown in figure~\ref{fig:depend}.
One difference from a Haskell version is the type of \te{interp}.
Dependent function types are written ``\te{(x~::~t)~->~s}'', where the
variable \te{x} may occur in \te{s}.  In this way the result type
of a function can depend on the value of the argument.  A second
difference from Haskell is manifested in the function \te{TypeOf}.  This function
computes a type.  The type of types is written ``\te{\#}'' in Cayenne and
it is treated mostly like other types.

\begin{figure*}
\begin{verbatim}
interp :: (e :: Expr) -> TypeOf e
interp (EBool b) = b
interp (EInt i)  = i
-- EAdd doesn't work

TypeOf :: Expr -> #
TypeOf (EBool _) = Bool
TypeOf (EInt _) = Int
TypeOf (EAdd _ _) = Int
\end{verbatim}
\caption{A dependently typed interpreter.}
\label{fig:depend}
\end{figure*}

This interpreter is well typed, but the \te{EAdd} case is missing.
The reason is that the expression ``\te{interp e1 + interp e2}''
is not well typed.  The type of \te{interp} states that it sometimes
returns \te{Int} and sometimes \te{Bool}, but we cannot know that
both (or any) operands will be of type \te{Int}.  Something more
is needed; when interpreting an expression we need some tangible
evidence that it is actually well typed, otherwise we cannot know
that the \te{EAdd} clause cannot fail.

Before we add this we will make the language more interesting by
adding some more cases to the expression type, figure~\ref{fig:emiddle},
and by formalizing the typing rules, table~\ref{fig:tmiddle}.
The typing rules are exactly what you can expect for this little
language.

\begin{figure*}
\begin{verbatim}
data Expr = EBool Bool | EInt Int 
          | EAdd Expr Expr | EAnd Expr Expr | ELE Expr Expr
\end{verbatim}
\caption{A slightly extended expression type.}
\label{fig:emiddle}
\end{figure*}

\begin{figure*}
\begin{tabular}{ccc}
\ruleh{}
    {\mbox{}}
    {\hastype{i}{int}}
\rsep
&
\ruleh{}
    {\mbox{}}
    {\hastype{b}{bool}}
\rsep
&
\\
\ruleh{}
    {\hastype{e_1}{int} \seph \hastype{e_2}{int}}
    {\hastype{e_1 + e_2}{int}}
&
\ruleh{}
    {\hastype{e_1}{bool} \seph \hastype{e_2}{bool}}
    {\hastype{e_1 \&\& e_2}{bool}}

\ruleh{}
    {\hastype{e_1}{int} \seph \hastype{e_2}{int}}
    {\hastype{e_1 <= e_2}{bool}}
\end{tabular}
\caption{Typing rules for the slightly extended expression type.}
\label{fig:tmiddle}
\end{figure*}

\section{The well typing predicate}
We want to capture the typing rules in such a way that we can
know when an expression is well typed.  One way to do this
to encode the typing rules as a predicate.  A predicate is
encoded as a function that yields a type.  The trick of the
encoding is that the type it yields should be empty when the
predicate is false, and non-empty when the predicate is true.
All of this is, of course, part of the Curry-Howard isomorphism.

\tmp{More about predicates}

The typing rules have been encoded as predicate, \te{HasType}
in figure~\ref{fig:mhastype},
together with some auxillary definitions.\footnote{As can be seen
from the example Cayenne does not place the same restrictions as Haskell
on type and constructor names.}
Since it is impossible to
pattern match on types in Cayenne we need to introduce a concrete
type, \te{Type}, to encode the possible types of an \te{Expr}.
The types \te{Truth} and \te{Absurd} are used to encode logical
truth and falsity respectively, but any non-empty and empty type
would do equally well. \tmp{More}

The \te{/\bs} type is simply pairs (with an unusual name).
It is used to represent conjunction.  Pairs represent conjunction
simply because if either of the component types of a pair is an
empty type then the whole pair is an empty type as well.

\begin{figure}
\begin{verbatim}
data Type = TBool | TInt

data Truth = truth

data Absurd =     -- the empty type, no constructors

data (/\) a b = (&) a b

HasType :: Expr -> Type -> #
HasType (EBool _)    (TBool) = Truth
HasType (EInt _)     (TInt)  = Truth
HasType (EAdd e1 e2) (TInt)  = HasType e1 TInt  /\ HasType e2 TInt
HasType (EAnd e1 e2) (TBool) = HasType e1 TBool /\ HasType e2 TBool
HasType (ELE  e1 e2) (TBool) = HasType e1 TInt  /\ HasType e2 TInt
HasType _            _       = Absurd
\end{verbatim}
\caption{The well typing predicate.}
\label{fig:mhastype}
\end{figure}

Why are we using a predicate instead of a boolean function?
The \te{HasType} function could as easily have been written to
yield a boolean instead of a type, but as we will soon see
it is really the type we need.  Furthermore, as we extend the
type system there will be cases when the predicate is easy
to write, but the boolean function could be very difficult
(or even impossible).

A complete interpreter for our little language is presented in
figure~\ref{fig:imiddle}.  The interpreter is now more complicated.
It takes the expression to interpret, the expected type of the
expression, and an object in the well-typing predicate above.
The \te{Decode} function simply takes the encoded representation
of the type into its real type.

The well-typing predicate is an empty type when the expression
is not well typed.  This means that in, e.g., the first clause
of \te{interp} the \te{p} argument has type \te{Truth}.
The most interesting clause is the last one.  It is there to make
the patterns exhaustive, without it the Cayenne compiler would
complain that not all cases were covered in \te{interp}.  In this
clause the type of the argument \te{p} must be \te{Absurd}.
This can be easily checked by examining the \te{HasType} predicate;
the cases where it is not \te{Absurd} were covered by the
other clauses in \te{interp}.  This means that if we reach the
last clause of \te{interp} then \te{p} must be an element of
the empty type, but this is of course absurd.  The absurdity
of it is handled by the case analysis of \te{p}.  The case expression
has no arms, since the empty type has no constructors; still
all constructors are covered (since there are none) and all the
right hand sides of the case have the same type as we want (\te{Decode t}).

We can now see the reason for having the \te{HasType} function
be a predicate.  This way the ill-typed case ends up as
a clause in the interpreter that has an argument in the
empty type, and which can thus never be reached.  The fact that
\te{p} is an element of the empty type makes it possible to
construct an expression of the right type on the right hand side
of the last clause.  A boolean value would have been of no use,
and we would have had to resort to calling, e.g., the \te{error}
function, despite the fact that the clause can never be reached.\footnote{
A better type system might be able to determine that the last
clause cannot be reached and can therefore be omitted, 
see eg.\ \cite{coquand:patternmatching}.}

Note how the pair type used for the operators is taken apart
and the constituent proofs are passed along in the recursive
calls to \te{interp}.  It is easily verified that they actually
have the right type by comparing with the definition of \te{HasType}.

\begin{figure}
\begin{verbatim}
interp :: (e :: Expr) -> (t :: Type) -> HasType e t -> Decode t
interp (EBool b)    (TBool) p         = b
interp (EInt  i)    (TInt)  p         = i
interp (EAdd e1 e2) (TInt)  (p1 & p2) = interp e1 TInt  p1 +  interp e2 TInt p2
interp (EAnd e1 e2) (TBool) (p1 & p2) = interp e1 TBool p1 && interp e2 TBool p2
interp (ELE  e1 e2) (TBool) (p1 & p2) = interp e1 TInt  p1 <= interp e2 TInt p2
interp _            _       p         = case p of { }

Decode :: Type -> #
Decode (TBool) = Bool
Decode (TInt)  = Int
\end{verbatim}
\caption{An interpreter for the slightly extended expression type.}
\label{fig:imiddle}
\end{figure}

\section{Adding variables}
The interpreted language in the previous sections is very simple.
We will now extend it to $\lambda$-calculus by adding variables,
application, and $\lambda$-expressions.  The $\lambda$s will be
explicitely typed to simplify things.

The extended expression type is shown in figure~\ref{fig:expr}.
The concrete type expressions and the decoding function are also
extended to handle the function type.

\begin{figure}
\begin{verbatim}
data Expr = EBool Bool | EInt Int 
          | EAdd Expr Expr | EAnd Expr Expr | ELE Expr Expr
          | EVar Symbol | ELam Type Symbol Expr | EAp Expr Expr

type Symbol = String

data Type = TBool | TInt | TArrow Type Type

Decode :: Type -> #
Decode (TBool) = Bool
Decode (TInt)  = Int
Decode (TArrow a b) = Decode a -> Decode b
\end{verbatim}
\caption{The expression type.}
\label{fig:expr}
\end{figure}

\subsection{The type system}
The typing rules must be extended with an environment (or
assumptions) to handle the variables.  These rules, figure~\ref{fig:rules},
are the standard ones.

\begin{figure}
\begin{tabular}{ccc}
\ruleh{}
    {\mbox{}}
    {\judge{\Gamma}{\hastype{i}{int}}}
\rsep
&
\ruleh{}
    {\mbox{}}
    {\judge{\Gamma}{\hastype{b}{bool}}}
\rsep
&
\\
\ruleh{}
    {\judge{\Gamma}{\hastype{e_1}{int}} \seph \judge{\Gamma}{\hastype{e_2}{int}}}
    {\judge{\Gamma}{\hastype{e_1 + e_2}{int}}}
\rsep
&
\ruleh{}
    {\judge{\Gamma}{\hastype{e_1}{bool}} \seph \judge{\Gamma}{\hastype{e_2}{bool}}}
    {\judge{\Gamma}{\hastype{e_1 \&\& e_2}{bool}}}
\rsep
&
\ruleh{}
    {\judge{\Gamma}{\hastype{e_1}{int}} \seph \judge{\Gamma}{\hastype{e_2}{int}}}
    {\judge{\Gamma}{\hastype{e_1 <= e_2}{bool}}}
\rsep
\\
\ruleh{}
    {\mbox{}}
    {\judge{\Gamma,\hastype{x}{t}}{\hastype{x}{t}}}
\rsep
&
\ruleh{}
    {\judge{\Gamma}{\hastype{f}{s \fun t}} \seph \judge{\Gamma}{\hastype{a}{s}}}
    {\judge{\Gamma}{\hastype{f\;a}{t}}}
\rsep
&
\ruleh{}
    {\judge{\Gamma,\hastype{x}{s}}{\hastype{e}{t}}}
    {\judge{\Gamma}{\hastype{\lam x:s. e}{s \fun t}}}
\end{tabular}
\caption{Typing rules for expressions.}
\label{fig:rules}
\end{figure}

The \te{HasType} predicate (see figure~\ref{fig:hastype}) must be extended with an environment, \te{g},
as well.  The environment is represented simply by a function from
symbols to types, and the function \te{extendT} is used to add
a binding.  We assume that the environment is a total function
(it can easily be made total).

The \te{EVar} clause simply looks up the variable in the type environment
and ``checks'' that it is equal to the type it should have.  The \te{EqType}
predicate is used to test if two types are equal.  It is written in the
obvious way, but yields a type instead of the usual boolean result.

The \te{ELam} clause follows the typing rule closely and checks that the
body has the right type in the extended environment.

The most interesting clause is \te{EAp}.  According to the rule an application
is well typed if there is some type $s$, that the argument has, which makes
the types fit together.  The type $s$ is not present in the conclusion of the
rule, which means that it is not present on the left hand side of \te{HasType}.
Instead, it is a component of a record type on the right hand side.
The \te{sig} keyword simply indicates a record type; it is followed by
the signatures for its components.
The record
has three parts, two of them correspond to the two premises (and could be represented
by a pair).  The first component of the record is the type \te{s} which makes the
other two components true.  The use of a record type corresponds to using
an existential quantifier. \tmp{MORE}

Note how the definition of the \te{HasType} predicate follows the typing rules
closely. The conclusions correspond to patterns in the predicate, and
the premises are the right-hand-side expressions, using \te{Truth} for the
axioms. The \te{Absurd} clause expresses that there are no other rules in
the system but these. There are some minor differences, though:

\begin{itemize}

\item Cayenne does not allow variables to occur more than once in a
  pattern, but in the rule for $\lambda$-expressions, the type $s$
  occurs twice. The \te{ELam} clause encodes this by using two
  variables, and an additional premise stating that these are
  equal. The same goes for the variable clause.

\item We need to be explicit about variables introduced in the
  premises, such as $s$ in the \te{EAp} clause.

\end{itemize}

\begin{figure}
\begin{verbatim}
type TEnv = Symbol -> Type
extendT :: TEnv -> Symbol -> Type -> TEnv

HasType :: TEnv -> Expr -> Type -> #
HasType g (EBool _)    (TBool) = Truth
HasType g (EInt _)     (TInt)  = Truth
HasType g (EAdd e1 e2) (TInt)  = HasType g e1 TInt  /\ HasType g e2 TInt
HasType g (EAnd e1 e2) (TBool) = HasType g e1 TBool /\ HasType g e2 TBool
HasType g (ELE  e1 e2) (TBool) = HasType g e1 TInt  /\ HasType g e2 TInt
HasType g (EVar x)     t       = EqType (g x) t
HasType g (EAp f a)    t       = 
    sig s  :: Type
        pf :: HasType g f (TArrow s t)
        pa :: HasType g a s
HasType g (Lam x s' e) (TArrow s t) = 
    HasType (extendT g x s) e t /\ EqType s' s
HasType _ _            _       = Absurd

EqType :: Type -> Type -> #
EqType (TInt) (TInt) = Truth
EqType (TBool) (TBool) = Truth
EqType (TArrow a1 r1) (TArrow a2 r2) = EqType a1 a2 /\ EqType r1 r2
EqType _ _ = Absurd
\end{verbatim}
\caption{The well typing predicate.}
\label{fig:hastype}
\end{figure}

\subsection{Representing the environment}
The interpreter for the full language needs to use an environment to
look up the values of variables.  The representation of this environment
is not totally obvious since the variables can have different types.
This problem can be solved by using both a type environment and a
value environment.  The type of the latter can then be expressed
using the former, as shown in figure~\ref{fig:env}.
As can be seen from this definition the return type of the
environment lookup depends on the value (encoding a type)
found in the type environment.

The implementation of both the type and value environments are 
straightforward, but the type of \te{extendV} is rather complicated.
The function \te{extendV} takes not only the expected arguments \te{r}, \te{x}, and \te{v},
but also the type environment, \te{g}, and the type of the value, \te{t}.
Without these it would be impossible to express the type since
the \te{VEnv} type is indexed by the type environment.
The \te{IF} function used in \te{entendV} is the if function that
allows the two branches to have different types.

The \te{extendV} function really has the most amazing type of any
function in this example, so let us examine why it is type correct
a little closer.  The type of the \te{IF} function is
\begin{verbatim}
 IF ::
   (tx :: #) |-> (ty :: #) |-> 
   (b :: Bool) -> tx -> ty -> case b of (True) -> tx; (False) -> ty
\end{verbatim}

The function arrow \te{|->} indicates that the argument does not
normally have to be given; the type checker deduces it instead.

The variable \te{v} has type ``\te{Decode t}'' and the expression
``\te{r x}'' has type ``\te{Decode (g x)}'' according to the type
of \te{r}.
So looking at the right hand side of \te{extendV} we can see that it has
type ``\te{(x' :: Symbol) -> case x == x' of (True) -> Decode t; (False) -> Decode (g x)}''.

According to the signature the type should be
``\te{VEnv (extendT g x t)}'', expand this and we get
``\te{(x :: Symbol) -> Decode (if (x == x') t (g x))}'', furthermore, 
expand the definition of \te{if} and we get
``\te{(x :: Symbol) -> Decode (case x == x' of (True) -> t; (False) -> g x}''.
Using the definition of \te{Decode} (which is really a case expression)
and the rule of commuting case expressions the type checker can verify that
this expression is indeed equal to the inferrred type above.

\begin{figure}
\begin{verbatim}
type VEnv (g :: TEnv) = (x :: Symbol) -> Decode (g x)

-- implementation of the environments
emptyT :: TEnv
emptyT = \ (s :: Symbol) -> TInt  -- make it total

extendT :: TEnv -> Symbol -> Type -> TEnv
extendT g x t = \ (x' :: Symbol) -> if (x == x') t (g x)

emptyV :: VEnv emptyT
emptyV = \ (x :: Symbol) -> 0     -- make it total

extendV :: (g :: TEnv) ->
           (r :: VEnv g) ->
           (x :: Symbol) ->
           (t :: Type) ->
           (v :: Decode t) ->
           (VEnv (extendT g x t))
extendV g r x t v = \ (x' :: Symbol) -> IF (x == x') v (r x)
\end{verbatim}
\caption{Representing the environments.}
\label{fig:env}
\end{figure}

\section{The interpreter}
We now have all the building blocks to make the complete interpreter,
figure~\ref{fig:interp}.  The application and $\lambda$ clauses simply
end up as an application and a $\lambda$-expression respectively.
For the application clause it is simply a matter of calling the
interpreter recursively with the right arguments; note the use of
\te{p.s} which is the type of the argument, \te{a}.  The type
of the argument is not apparant from the value or type argument to
\te{interp}, but can be found in the ``proof''.  In the $\lambda$ clause
the type and value environments must be extended in the obvious way.

The most complicated clause is really the variable access.
The variable is looked up by ``\te{(r x)}''.  This expression has
the (encoded) type ``\te{(g x)}'', but the expected return type is \te{t}.
These two types are not obviously equal, but \te{p} is a proof
that they actually are equal since it has type ``\te{EqType (g x) t}''.
The function \te{convert} converts a value belonging to one
type into the same value of another type given a proof that
the types are actually equal.  The \te{convert} functions
is thus nothing but an identity function and its only purpose
is to placate the type checker.

\begin{figure}
{\small
\begin{verbatim}
interp :: (e :: Expr) -> (t :: Type) -> (g :: TEnv) -> 
          VEnv g -> HasType g e t -> Decode t
interp (EBool b)    (TBool) g r p         = b
interp (EInt  i)    (TInt)  g r p         = i
interp (EAdd e1 e2) (TInt)  g r (p1 & p2) = interp e1 TInt  g r p1 +  interp e2 TInt  g r p2
interp (EAnd e1 e2) (TBool) g r (p1 & p2) = interp e1 TBool g r p1 && interp e2 TBool g r p2
interp (ELE  e1 e2) (TBool) g r (p1 & p2) = interp e1 TInt  g r p1 <= interp e2 TInt  g r p2
interp (Var x)      t       g r p         = convert (g x) t p (r x)
interp (App f a)    t       g r p         = (interp f (TArrow p.s t) g r p.pf) 
                                            (interp a p.s            g r p.pa)
interp (Lam x _ e)  (TArrow s t) g r (p & _) = 
     \ (v :: Decode s) -> interp e t (extendT g x s) (extendV g r x s v) p
interp _            _       g r p         = case p of { }

-- An identity function.
convert :: (t :: Type) -> (t' :: Type) -> EqType t t' -> Decode t -> Decode t'
\end{verbatim}
}
\caption{The interpreter.}
\label{fig:interp}
\end{figure}

\tmp{MORE}

\section{Type checking}
Whenever the \te{interp} function is called it has to be given
a value that represents the proof that the expression is well typed.
This can, of course, be constructed by hand for any particular
(well typed) expression.  On the other hand, if the expression comes
from a parser we would like it to be generated dynamically.
In this section we will take a look at a type checker that given
an expression will either return an indication that it is ill typed,
or the type of the expression together with the proof that it is
well typed.  Its type is shown in figure~\ref{fig:tchecktype}.

\begin{figure}
\begin{verbatim}
tcheck :: (e :: Expr) -> (g :: TEnv) -> 
          Maybe (sig { t :: Type; p :: HasType g e t })
\end{verbatim}
\caption{The type of the type checker.}
\label{fig:tchecktype}
\end{figure}

\tmp{MORE}

\section{Putting it all together}
Assuming that we have a parser and a printing function we can now
put all the pieces together into a program that take a string,
parses it, type checks the expression, interprets it, and finally
prints the result.
All of this is show in
figure~\ref{fig:pp}.

As shown the interpreter assumes that all free variables have value \te{0},
but it would be easy to add a check that there are no free variables
on the top level.  The type of this check can be refined to include
a proof that there are no free variables, but this is left as an exercise
for the reader.

\begin{figure}
\begin{verbatim}
parse :: String -> Maybe Expr
parse s = ...

print :: (t :: Type) -> Decode t -> String
print (TInt) i = System$Int.show i
print (TBool) b = System$Bool.show b
print (TArrow _ _) f = "<<function>>"

run :: String -> String
run s =
    case parse s of
    (Nothing) -> "Parse error!"
    (Just e) ->
        case tcheck e emptyT of
        (Nothing) -> "Type error!"
        (Just tp) -> print tp.t (interp e tp.t emptyT emptyV tp.p)
\end{verbatim}
\caption{A complete program.}
\label{fig:pp}
\end{figure}



\section{Efficiency}
The interpreter we have presented is more efficient than one
using a tagged representation because it can avoid the repeated
tagging and untagging.  On the other hand, our version of the
interpreter has three extra arguments (the result type, the
type environment, and the well typing proof).  These extra
arguments are really superflous from a computational point of
view. \cite{carlsson:disposable} describes a method where the
Cayenne programmer uses type annotations to flag expressions and
arguments that are only needed during type checking. The compiler is
then able to remove all arguments except the expression to interpret, and the
value environment. For some expressions, the resulting interpreter runs
considerably faster than a corresponding interpreter written in
Haskell, due to the lack of tagging and untagging.

\tmp{More on how to get rid of them}

A further potential source of inefficiency is the \te{convert}
function.  It is really an identity function, but the most
straightforward implementation of it will result in an
function the will completely traverse the two \te{Type}
arguments before returning the value.  This problem can
also be remedied.

\tmp{More on how to solve this problem}

\section{Conclusion}
Using dependent types it is possible to write an interpreter that
avoids tagging, but there are several other benefits.
\begin{itemize}
\item 
The typing rules were almost trivially captured as the \te{HasType} predicate
and since the correspondance is so close it is unlikely that the definition
is wrong.
\item
The type checker must be faithful to the typing rules since it delivers
a type for the expression that is correct according to \te{HasType}.
\item
We do not know if the interpreter respects the operational semantics of
the language (since the semantics has not been formalized), but we know
that the interpreter respects the types of the language since it
uses the \te{HasType} predicate.
\end{itemize}

This means that certain kind of errors in the type check and/or interpreter,
e.g., claiming that integer constants have type bool, will be caught by
the type checker.

\tmp{More}

\section{Acknowledgments}
Thanks to Jim Hook for getting one of us (Lennart) started on the interpreter and
for showing how to handle environments for a typed interpreter.
Thanks also to Jessica who helped improving the English of this paper.

\bibliography{/usr/local/lib/pmgrefs,interp}
\bibliographystyle{alpha} 

\end{document}
