%\batchmode
\nonstopmode

%%% \documentclass{acmconf}     %% ICFP
\documentclass{llncs}[12pt]             %% Portugal

\usepackage[latin1]{inputenc}
\usepackage[T1]{fontenc}

\usepackage{proof}
%\pagestyle{empty}
%\thispagestyle{empty}

\newcommand\bs{\char '134}
%\newcommand\notyet[1]{#1}
\newcommand\notyet[1]{}
\newcommand\com[1]{}
\newcommand\nocom[1]{#1}
\newcommand\tmp[1]{{\sf #1}}

%\leftmargin -2cm
%\leftmargini 0cm
%\leftmarginv 0cm
%\leftmarginvi 0cm
%\textwidth 16cm
%\topmargin -1.0cm
%\textheight 24.5cm
%\setlength\leftmargin {-2cm}

\newcommand{\nt}[1]{{\it #1}}
\newcommand{\te}[1]{{\tt #1}}
\newcommand{\spp}{\mbox{~~}}

\newcommand{\ruleh}[3]{\infer [$\sf \scriptsize #1$] {#3} {#2}}
\newcommand{\seph}{ & }
\newcommand{\judge}[2]{#1 \vdash #2}
\newcommand{\judget}[2]{#1 & \vdash & #2}
\newcommand{\hastype}[2]{#1 \in #2}
%\newcommand{\rsep}{\vspace*{5mm}}
\newcommand{\rsep}{\medskip\medskip}
\newcommand{\dc}{$\te{::}$}
\newcommand{\dt}{$\te{.}$}
\newcommand{\sm}{$\te{;}$}
\newcommand{\teq}{$\te{=}$}
\newcommand{\fun}{\rightarrow}
\newcommand{\lam}{\lambda}
\newcommand{\eq}[2]{#1 \approx #2}
\newcommand{\hash}[1]{\;$\te{\#}$_{#1}}
\newcommand{\keyw}[1]{$\te{#1}$\;}

\newcommand{\gram}[2]{\mbox{} \par \noindent {\it #1} ::= #2}
\newcommand{\nterm}[1]{{\it #1}}
\newcommand{\term}[1]{``{\tt #1}''}
\newcommand{\many}[1]{\{ #1 \}}
\newcommand{\opt}[1]{[ #1 ]}
\newcommand{\alt}{~|~}

\newcommand{\subst}[3]{#1[#2 \mapsto #3]}

\def\mapstou{\mapstochar\rightharpoonup}
\newcommand{\eval}[2]{#1 & $\longmapsto$ & #2 \\}
\newcommand{\evalu}[2]{#1 & $\mapstou$ & #2 \\}
\newcommand{\ieval}[2]{$ #1 \longmapsto #2 $}
\newcommand{\ievalu}[2]{$ #1 \mapstou #2 $}
\newcommand{\imeval}[2]{$ #1 \longmapsto^* #2 $}
\newcommand{\imevalu}[2]{$ #1 \mapstou^* #2 $}


\author{
Lennart Augustsson
}
%%\affiliation{
\institute{
{ Department of Computing Sciences}\\
{ Chalmers University of Technology}\\
{ S-412 96 G{\"o}teborg, Sweden}\\
Email: {\tt augustss@cs.chalmers.se}\\
WWW: {\tt http://www.cs.chalmers.se/\~{}augustss}\\
}
\date{}
\title{
Equality proofs in Cayenne
}

\begin{document}
\maketitle

\begin{abstract}
We introduce a simple syntactical extension to Cayenne to make equality proofs readable.
\end{abstract}

%% \copyrightspace %% ACM

\section{Introduction}
Equality proofs are important for functional programs, since they are
encouraged by equational reasoning.
Unfortunately, type theory style equational proofs are tedious and look very ugly.

Let us illustrate this with an example.  We wish to prove that list
append is associative.  The function has the following definition:
\begin{verbatim}
Nil      ++ ys = ys
(x : xs) ++ ys = x : (xs ++ ys)
\end{verbatim}
Now we want to establish that \te{(xs++ys)++zs = xs++(ys++zs)}.  With normal
notation we do this with an inductive proof\footnote{Assuming finite lists, and no $\perp$.} as follows.

\begin{tabular}{lcl}
Base case: \\
\te{~~(Nil++ys)++zs}	& \te{~=~} & \nt{~~~~~by first clause of ++} \\
\te{=~ys++zs}		& \te{~=~} & \nt{~~~~~by first clause of ++, reversed} \\
\te{=~Nil++(ys++zs)}	\\
\\
Inductive step: \\
\te{~~((x:xs)++ys)++zs}	& \te{~=~} & \nt{~~~~~by second clause of ++} \\
\te{=~(x:(xs++ys))++zs}	& \te{~=~} & \nt{~~~~~by second clause of ++} \\
\te{=~x:((xs++ys)++zs)}	& \te{~=~} & \nt{~~~~~by induction hypothesis} \\
\te{=~x:(xs++(ys++zs))}	& \te{~=~} & \nt{~~~~~by second clause of ++, reversed} \\
\te{=~(x:xs)++(ys++zs)} \\
\end{tabular}

Now compare this to the same proof in type theory style.  We assume that we have an \te{Id} type for
equality;\footnote{Which seems to be out of fashion these days.} see appendix \ref{app:Id}.
This proof uses
Cayenne/Agda syntax,\cite{augustsson:cayenne}, and assumes there are no hidden arguments.

\begin{verbatim}
(++) :: (a :: #) -> List a -> List a -> List a
(++) a (Nil) ys = ys
(++) a (x : xs) ys = (:) a x ((++) a xs ys)

appendAssocP :: (a :: #) -> (xs, ys, zs :: List a) -> 
        Id (List a) ((++) a ((++) a xs ys) zs) ((++) a xs ((++) a ys zs))
appendAssocP a (Nil) ys zs = 
        refl (List a) ((++) a ys zs)
appendAssocP a (x:xs) ys zs =
        subst (List a) (List a) 
              ((++) a ((++) a xs ys) zs)
              ((++) a xs ((++) a ys zs))
              (\ (ws :: List a) -> x : ws)
              (appendAssocP a xs ys zs)
\end{verbatim}

This is a total mess, and penetrating its meaning is only recommended for
masochists.  One reason for the mess is all the extra type arguments.  As a first step
towards readability
we allow hidden arguments (as Cayenne does), and use the infix symbol \te{===}
for the Id type.

\begin{verbatim}
(++) :: (a :: #) |-> List a -> List a -> List a
(Nil)    ++ ys = ys
(x : xs) ++ ys = x : (xs ++ ys)

appendAssocP :: (a :: #) |-> (xs, ys, zs :: List a) -> 
        (xs++ys)++zs === xs++(ys++zs)
appendAssocP (Nil) ys zs = 
        refl (ys++zs)
appendAssocP (x:xs) ys zs =
        subst (\ (ws :: List a) -> x : ws) (appendAssocP xs ys zs)
\end{verbatim}

The proof we have now is more readable, but it is quite tedious to construct
and read.  This is not because of any technical problems.  This is, in fact,
exactly the same proof as presented informally above, but we would be hard pressed
see that.

In contrast, the new notational device that we use in Cayenne to make the proof readable will
yield the following proof.

\begin{verbatim}
appendAssocP :: (a :: #) |-> (xs, ys, zs :: List a) -> 
        (xs++ys)++zs === xs++(ys++zs)
appendAssocP (Nil) ys zs =
        (Nil ++ ys) ++ zs               ={ DEF }=
        ys ++ zs                        ={ DEF }=
        Nil ++ (ys ++ zs)
appendAssocP (x:xs) ys zs =
        ((x : xs) ++ ys) ++ zs          ={ DEF }=
        (x : (xs ++ ys)) ++ zs          ={ DEF }=
        x : ((xs ++ ys) ++ zs)          ={ appendAssocP xs ys zs }=
        x : (xs ++ (ys ++ zs))          ={ DEF }=
        (x : xs) ++ (ys ++ zs)
\end{verbatim}

The notation is now quite close to the informal proof.  The notation is very simple.
It is a number of expressions that are equal, separated by the reason why they are equal.
If the reason says \te{DEF} it means that they are equal because of the definitional
equations; if it is an expression, then this expression is the proof object
that proves that the subexpressions that differ between the two expressions are equal.

\section{Equality proof notation}
The notation is almost entirely syntactic sugar.  There is a simple translation
from an equality proof into a normal term.
An equality proof has the following form:

\begin{tabular}{ll}
$exp_1$ & \te{=\{} $why_1$ \te{\}=} \\
$exp_2$ & \te{=\{} $why_2$ \te{\}=} \\
\ldots& \\
$exp_n$ & \te{=\{} $why_n$ \te{\}=} \\
$exp_{n+1}$ \\
\end{tabular}

The first expression, $exp_1$, is used to prove that it is equal to itself
using reflexivity.  Given the proof from an initial segment of an 
equality chain, it is easy to obtain the next proof.  If the reason,
$why_k$, is \te{DEF} it simply means that the same equality proof can be used 
even when the next step is added since the type checker can handle
definitional equalities.
If the reason is a proof object, it is the proof that a subpart of $exp_k$
is equal to a corresponding part of $exp_{k+1}$.  The rest of the two expressions
must be identical.  Using substitutivity on this proof we can obtain a proof
that $exp_k$ is equal to $exp_{k+1}$.  This proof is then combined with the
proof from the preceding steps in the chain using transitivity.

The two subexpressions that a given proof object concerns can be explicitly
marked by enclosing the subexpression in \te{[]} (in the first of the two involved
expressions).  If no subexpression is
marked, the type checker will use the smallest subexpressions that make the rest
of the expressions identical.

It is almost always sufficent to leave out the \te{[]} marks and let the
smallest changed be derived automatically, but sometimes this is not
correct.  For example, the expressions ``\te{abs(abs x)}'' is equal to
``\te{abs x}'', but this is {\em not} because we have a proof that
``\te{abs x}'' is equal to ``\te{x}'', so the smallest subexpression that
differs is not the right thing to choose in this case.

The extension to the syntax and the translation can be formalized as follows.  

The extension to the grammar:

\gram{exp}{\term{[} \nterm{exp} \term{]}}
\gram{exp}{\nterm{eqchain}}

\gram{eqchain}{\nterm{exp}}
\gram{eqchain}{\nterm{eqchain} \term{=\{} \term{DEF} \term{\}=} \nterm{exp}}
\gram{eqchain}{\nterm{eqchain} \term{=\{} \nterm{exp} \term{\}=} \nterm{exp}}

Note that this is not the concrete grammar used for parsing.  An equality
proof must have at least one ``\te{=\{~\}=}'' to make sense.

The bracketed expression that indicates what part of the term the
equality proof operates on is taken from the expression that precedes the
``\te{=\{~\}=}''.

Example:
\begin{verbatim}
    f[x]  ={     h  ::   x === y   }= 
    [f y] ={     h' :: f y === g y }= 
     g[y] ={ symm h ::   y === x   }=
     g x
\end{verbatim}

The translation:

\newcommand{\eqt}[1]{{\cal E}\lbrack\!\lbrack\:#1\:\rbrack\!\rbrack}

\[
\begin{array}{lcl}
\eqt{e} & = & \te{refl}\; e \\
\eqt{c\; \te{=\{~DEF~\}=}\; e} & = & \eqt{c}\; \te{::}\; \te{(}h(c)\; \te{===}\; e \te{)} \\
\eqt{c\; \te{=\{}\;p\;\te{\}=}\; C[e]} & = & \te{trans}\; \eqt{c}\; 
	\te{(subst~(\bs~}x\te{~->~} C[x] \te{)}\; p \te{)} \\
\end{array}
\]
\[
\begin{array}{lcl}
h(e) & = & e \\
h(c\; \te{=\{~...~\}=}\; e) & = & h(c)
\end{array}
\]

The \te{DEF} steps can, of course, be left out of an equality proof since
they do not contribute to the proof object, but they can increase the
clarity of a proof.

\section{Possible extensions}
\subsection{Symmetry}
The notation handles substitutivity and transitivity, but not symmetry of
equality.  It would be easy to extend the type checker to try a given
$why$ expression with and without symmetry applied, and pick the one that
succeeds, but we have not implemented this.

\subsection{Proof simplification}
The equality proof obtained by the translation described above is not
always as simple as it could be.
The first time that transitivity is used in a proof we know that the
first proof will always be a proof with reflexivity, so it is not really
necessary to use transitivity; the second proof to \te{trans} suffices
as the whole proof.  Furthermore, sometimes the function argument used
to \te{subst} is the identity function, so there is really no need to use
\te{subst} at all.

It is easy to add these simplifications to the translation process,
but it does not really matter since the proof objects are not used anyway.

\section{Acknowledgments}
Thanks to Jessica for proof reading.

\bibliography{/usr/local/lib/pmgrefs,interp}
\bibliographystyle{alpha} 

\appendix
\section{The equality type}
\label{app:Id}
\begin{verbatim}
(===) :: (a :: #) |-> a -> a -> #

refl :: (a :: #) |-> (x :: a) |-> x === x

symm :: (a :: #) |-> (x, y :: a) |->
        x === y -> y === x

trans :: (a :: #) |-> (x, y, z :: a) |-> 
        x === y -> y === z -> x === z

congr :: (a :: #) |-> (P :: a -> #) -> (x, y :: a) -> 
        x === y -> P x -> P y

subst :: (a, b :: #) |->  (x, y :: a) |-> (f :: a -> b) ->
        x === y -> f x === f y
\end{verbatim}

\section{Complete proof of \te{rev (rev xs) = xs}}
{
\small
\begin{verbatim}
module example$RevEq =
open System$List use * in
open System$Id use (===), symm in

struct

concrete
(++) :: (a :: #) |-> List a -> List a -> List a
(++) (Nil) ys = ys
(++) (x : xs) ys = x : (xs ++ ys)

appendNilP :: (a :: #) |-> (xs :: List a) -> 
        xs ++ Nil === xs
appendNilP |a (Nil) =
        Nil|a ++ Nil                          ={ DEF }=
        Nil
appendNilP |a (x : xs) =
        (x:xs) ++ Nil                         ={ DEF }=
        x:(xs ++ Nil)                         ={ appendNilP xs }=
        x:xs                                  ={ DEF }=
        Nil ++ (x:xs)

appendAssocP :: (a :: #) |-> (xs, ys, zs :: List a) -> 
        (xs++ys)++zs === xs++(ys++zs)
appendAssocP |a (Nil) ys zs =
        (Nil|a ++ ys) ++ zs                   ={ DEF }=
        ys ++ zs                              ={ DEF }=
        Nil ++ (ys ++ zs)
appendAssocP |a (x:xs) ys zs =
        ((x : xs) ++ ys) ++ zs                ={ DEF }=
        (x : (xs ++ ys)) ++ zs                ={ DEF }=
        x : ((xs ++ ys) ++ zs)                ={ appendAssocP xs ys zs }=
        x : (xs ++ (ys ++ zs))                ={ DEF }=
        (x : xs) ++ (ys ++ zs)

-----

concrete
rev :: (a :: #) |-> List a -> List a
rev (Nil) = Nil
rev (x : xs) = rev xs ++ (x : Nil)

revAppendP :: (a :: #) |-> (xs, ys :: List a) -> 
        rev (xs++ys) === rev ys ++ rev xs
revAppendP |a (Nil) ys =
        rev (Nil|a ++ ys)                     ={ DEF }=
        rev ys                                ={ symm (appendNilP (rev ys)) }=
        rev ys ++ Nil                         ={ DEF }=
        rev ys ++ rev (Nil|a)
revAppendP |a (x:xs) ys =
        rev ((x : xs) ++ ys)                  ={ DEF }=
        rev (x : (xs ++ ys))                  ={ DEF }=
        rev (xs ++ ys) ++ (x:Nil)             ={ revAppendP xs ys }=
        (rev ys ++ rev xs) ++ (x:Nil)         ={ appendAssocP (rev ys) (rev xs) (x:Nil) }=
        rev ys ++ (rev xs ++ (x:Nil))         ={ DEF }=
        rev ys ++ (rev (x:xs))

revRevP :: (a :: #) |-> (xs :: List a) -> 
        rev (rev xs) === xs
revRevP |a (Nil) =
        rev (rev (Nil|a))                     ={ DEF }=
        rev (Nil|a)                           ={ DEF }=
        Nil
revRevP |a (x:xs) =
        rev (rev (x:xs))                      ={ DEF }=
        rev (rev xs ++ (x:Nil))               ={ revAppendP (rev xs) (x:Nil) }=
        rev (x:Nil) ++ rev (rev xs)           ={ revRevP xs }=
        rev (x:Nil) ++ xs                     ={ DEF }=
        (x:Nil) ++ xs                         ={ DEF }=
        x:xs
\end{verbatim}
}

\end{document}
