\nonstopmode

\documentclass[17pt]{foils} % foils
\usepackage{alltt}
\usepackage{proof}

\LogoOff


\setlength{\parindent}{0pt}
\setlength{\parskip}{1ex}

\newcommand{\te}[1]{{\tt #1}}
\newcommand{\sss}[1]{\foilhead[-3ex]{{\hcolor{blue}#1}}}
\newcommand{\ssss}[1]{\foilhead{{\hcolor{blue}#1}}}
\newcommand{\sssss}[1]{\foilhead{\hcolor{blue}{#1}}}

\newcommand{\sub}[1]{{\bf #1}\medskip}

\newcommand{\hcolor}[1]{\color{#1}}

\newcommand{\fcolor}{\hcolor{green}} %% color of functions
\newcommand{\icolor}{\hcolor{green}} %% color of inlined code
\newcommand{\icol}[1]{{\hcolor{red}#1}} %% color of inlined code

\newcommand{\ncolor}{\hcolor{red}}
\newcommand{\headingcolor}{\hcolor{headingcolor}}
\newcommand{\dencolor}{\hcolor{red}}
\newcommand{\emmcolor}{\hcolor{red}}
\newcommand{\progcolor}{\hcolor{green}}

\newcommand{\varvalcolor}{\hcolor{red}}
\newcommand{\heapvalcolor}{\hcolor{red}}

\newcommand\bs{\char '134}  % A backslash character for \tt font

\title{
Cayenne -- stronger than Haskell
}
\author{ Lennart Augustsson
}
\date{ October 1997 }
\begin{document}
\maketitle

Cayenne is another functional language.  It has
\begin{itemize}
\item Haskell like syntax
\item dependent types
\item explicit polymorphism
\item powerful records
\item a Java-inspired module system
\end{itemize}

It does {\em not} have
\begin{itemize}
\item overloading
\item classes
\item decidable type checking
\item type deduction
\item correctness proofs (yet)
\end{itemize}

\newpage

\foilhead{The basic types:}

%Because of the dependent types the value and types expressions
%have the same syntax (unlike Haskell).

\begin{itemize}
\item Functions
\begin{alltt}
Term                       Type
\bs (x::t) -> e              t -> s
                           (x::t) -> s

f e

t -> s                     #

\end{alltt}
\item Data types
\begin{alltt}
C1@t                       data C1 | C2 Int | C3 a

case e of
(C1)   -> e1
(C2 i) -> e2
(C3 x) -> e3

data C1 | C2 Int | C3 a    #

\end{alltt}
Constructor names need not be unique and they live in
a separate name space.

\newpage
\item Record types
\begin{alltt}
struct \{ x1 :: t1 = e1;     sig \{ x1 :: t1;
         x2 :: t2 = e2 \}          x2 :: t2 \}

e.x1

sig \{ x1 :: t1; x2 :: t2 \} #

\end{alltt}
The definitions in the record can be mutually recursive.
Record labels need not be unique.

Record embellishments:
\begin{alltt}
struct                     sig
  x1 :: t1 = e1;             x1 :: t1
  private x2 :: t2 = e2


\end{alltt}
\begin{alltt}
struct                     sig
  x1 :: t1 = e1;             x1 :: t1;
  concrete x2 :: t2 = e2     x2 :: t2 = e2


\end{alltt}
There are also the keywords {\tt public} and {\tt abstract}
as the opposites of {\tt private} and {\tt concrete}.
Definitions of type {\tt \#} default to {\tt concrete}
other definitions to {\tt abstract}.

\end{itemize}

\newpage
\foilhead{Syntactic spices}
\begin{itemize}
\item Types can be left out on all bindings except recursive ones.

\item Let expressions
\begin{alltt}
let x1 :: t1
    x1 = e1
    x2 :: t2
    x2 = e2
in  e

\end{alltt}

\item Function definitions
\begin{alltt}
f :: Int -> Int -> Int     f :: Int -> Int -> Int
f x y = e                  f = \bs (x::Int) -> 
                                 (y::Int) -> e

\end{alltt}

\item Data type definitions
\begin{alltt}
data Bool = False | True   Bool :: #
                           Bool = data False | True
                           False :: Bool
                           False = False@Bool
                           True :: Bool
                           True = True@Bool

\end{alltt}

\newpage
\item Type definitions
\begin{alltt}
type T a = e               T :: # -> #
                           T = \ (a :: #) -> e

\end{alltt}

\item Open
\begin{alltt}
open e                     let r = e
use  x1::t1,x2::t2             x1 = r.x1
in   e'                        x2 = r.x2
                           in  e'
\end{alltt}

\item Infix operators

There are a few predefined infix operators.

\item Text inclusion

A few predefined texts, corresponding to importing the
standard prelude, can be included.

\end{itemize}

\newpage
\foilhead{Examples}

\begin{itemize}
\item The identity function
\begin{alltt}
Haskell                   Cayenne
id :: a -> a              id :: (a :: #) -> a -> a
id x = x                  id a x = x

id True                   id Bool True

\end{alltt}

\item Simple records
\begin{alltt}
let Coord = sig \{ x :: Int; y :: Int \}
    origin :: Coord
    origin = struct \{ x = 0; y = 0 \}
    manhattan :: Coord -> Int
    manhattan c = c.x + x.y
in  manhattan origin

\end{alltt}

\item Lists
\begin{alltt}
data List a = Nil | Cons a (List a)

\end{alltt}

\newpage
\item Dependent records
\begin{alltt}
struct
    data Bool = False | True
    not :: Bool -> Bool
    not (True)  = False
    not (False) = True
    if :: (a :: #) -> Bool -> a -> a -> a
    if a (True)  t e = t
    if a (False) t e = e
\end{alltt}
This has type
\begin{alltt}
sig
    Bool :: # = data False | True
    False :: Bool = False@Bool
    True :: Bool = True@Bool
    not :: Bool -> Bool
    if :: (a :: #) -> Bool -> a -> a -> a
\end{alltt}


\end{itemize}

\newpage

\foilhead{Modules}

\begin{alltt}
module foo$bar = e

\end{alltt}

A module is simply a named expression.  Module names live in a 
separate global name space.  Module identifiers always contain
a ``\$''.

A type, as well as {\tt concrete} can be given as well.
\begin{alltt}
module concrete foo$bar :: t = e

\end{alltt}

\begin{alltt}
module System$Bool = sig
    data Bool = False | True
    not :: Bool -> Bool
    not (True)  = False
    not (False) = True
    if :: (a :: #) -> Bool -> a -> a -> a
    if a (True)  t e = t
    if a (False) t e = e
\end{alltt}



\newpage
\begin{alltt}

module foo$bar = sig
tt :: System$Bool.Bool = System$Bool.True
ff :: System$Bool.Bool = System$Bool.False

module foo$baz = 
open System$Bool use Bool, not, if in
struct
not_if :: (a :: #) -> Bool -> a -> a -> a
not_if a c t e = if a (not c) t e

\end{alltt}

\newpage
\foilhead{Hidden arguments}
The type arguments are tedious.  But they can sometimes be left out!
We introduce a new type of functions, where the argument can
be left out.
\begin{alltt}
\bs (x::t) |-> e           (x::t) |-> s

f | e          -- explicit argument
f              -- hidden argument
\end{alltt}
Examples:
\begin{alltt}
let id :: (a :: #) |-> a -> a
    id x = x
in  id True

if :: (a :: #) |-> Bool -> a -> a -> a
if (True)  t e = t
if (False) t e = e

if True e1 e2
\end{alltt}
\mbox{~~~}or
\begin{alltt}
if |Int True 1 x
\end{alltt}

One necessary (but not sufficient) restriction that these functions
must obey is that the type ``{\tt (x::t) |-> s}'' is only well formed
if the variable ``{\tt x}'' occurrs free the expression ``{\tt s}''.

\newpage
\foilhead{Typechecking with dependent types}
The interesting part of typechecking is the case expression.

\begin{alltt}
case x of
    (True) -> 1
    (False) -> "Hello"
 :: (case x of
       (True) -> Int
       (False) -> String)
\end{alltt}

Keep track of the {\em value} of variables as well
as their type!  Use values when normalizing types.

\newpage
Check:
\begin{alltt}
case x of
    (True) -> 1
    (False) -> "Hello"
 :: (case x of
       (True) -> Int
       (False) -> String)


\end{alltt}

Check 1st arm:
\begin{alltt}
 1 :: (case x of
        (True) -> Int
        (False) -> String)    [x := True]
\end{alltt}
which reduces to
\begin{alltt}
 1 :: Int

\end{alltt}

Check 2nd arm:
\begin{alltt}
 "Hello" :: 
      (case x of
        (True) -> Int
        (False) -> String)    [x := False]
\end{alltt}
which reduces to
\begin{alltt}
 "Hello" :: String
\end{alltt}

\newpage
\foilhead{On stacks and queues}

\begin{alltt}
module concrete example$STACK = sig
Stack :: # -> #
empty :: (a :: #) |-> Stack a
push :: (a :: #) |-> a -> Stack a -> Stack a
pop :: (a :: #) |-> Stack a -> Stack a
top :: (a :: #) |-> Stack a -> a
isEmpty :: (a :: #) |-> Stack a -> System$Bool.Bool

\end{alltt}
\newpage
\begin{alltt}
module example$StackL :: example$STACK =
#include Prelude
struct
abstract type Stack a = List a

empty :: (a :: #) |-> Stack a
empty = Nil

push :: (a :: #) |-> a -> Stack a -> Stack a
push x xs = Cons x xs

pop :: (a :: #) |-> Stack a -> Stack a
pop (Nil) = error "pop"
pop (_ : xs') = xs'

top :: (a :: #) |-> Stack a -> a
top (Nil) = error "top"
top (x : _) = x

isEmpty :: (a :: #) |-> Stack a -> Bool
isEmpty (Nil) = True
isEmpty (_ : _) = False
\end{alltt}

\newpage
\begin{alltt}
module concrete example$QUEUE = sig
Queue :: # -> #
empty :: (a :: #) |-> Queue a
enqueue :: (a :: #) |-> a -> Queue a -> Queue a
dequeue :: (a :: #) |-> Queue a -> Queue a
first :: (a :: #) |-> Queue a -> a

\end{alltt}

\newpage
\begin{alltt}
module example$StackToQueue :: 
  example$STACK -> example$QUEUE =
  \bs (s :: example$STACK) ->
#include Prelude
open s use Stack, push, pop, top, isEmpty in
struct
abstract type Queue a = Stack a

empty :: (a :: #) |-> Queue a
empty = s.empty

enqueue :: (a :: #) |-> a -> Queue a -> Queue a
enqueue x xs = app xs x

dequeue :: (a :: #) |-> Queue a -> Queue a
dequeue xs = pop xs

first :: (a :: #) |-> Queue a -> a
first xs = top xs

private 
app :: (a :: #) |-> Stack a -> a -> Stack a
app xs y =
    if (isEmpty xs) 
       (push y s.empty)
       (push (top xs) (app (pop xs) y))
\end{alltt}
\newpage
\begin{alltt}
module example$usestack = 
#include Prelude
open example$StackToQueue example$StackL 
use enqueue, first, empty in
append
  (show (first (enqueue 1 (enqueue 2 empty))))
  "{\bs}n"

\end{alltt}


\newpage
\foilhead{Vectors with dependent types.}

We would like to have the length of a vector manifest in the
type to ensure that vectors are combined in the right way.

\begin{alltt}
VECTOR = sig
    type Vector (n :: Integer) a
    concatV :: (n :: Integer) |-> (m :: Integer) |-> 
               (a :: #) |-> Vector n a -> Vector m a -> 
               Vector (n + m) a
    zipWithV :: (n :: Integer) |->
                (a :: #) |-> (b :: #) |-> (c :: #) |->
                (a->b->c) -> Vector n a -> Vector n b -> 
                Vector n c
    mkVector :: (a :: #) |-> (n :: Integer) ->
                (Integer -> a) -> Vector n a

\end{alltt}

Example:
\begin{alltt}
let x = mkVector 5 id
    y = mkVector 2 id
    z = mkVector 3 id
    yz = concatV y z
    xyz = zipWithV (+) x yz

\end{alltt}

\newpage
A sample implementation with lists.
\begin{alltt}
module example$vector :: VECTOR =
#include Prelude
struct

abstract
type Vector (n::Integer) a  = List a

concatV xs ys = xs ++ ys

zipWithV f xs ys = zipWith f xs ys

mkVector n f = map f (enumFromTo 0 (n-1))

\end{alltt}

\newpage
\foilhead{The tautology function.}

We want a function that takes a boolean formula
represented as a function and tells us if this
formula is a tautology or not.

In an untyped language this can be written as:
\begin{alltt}
taut x = if isFunction x then
            taut (x True) && taut (x False)
         else
            x
\end{alltt}

To give this a type we need an extra argument
that tell how many arguments x needs.

\begin{alltt}
taut 0 x = x
taut (n+1) x = taut n (x True) && taut n (x False)
\end{alltt}

In Cayenne it looks like this

\begin{alltt}
data Nat = Zero | Succ Nat

taut :: (n::Nat) -> TautArg n -> Bool
taut (Zero) x = x
taut (Succ m) f = taut m (f True) && taut m (f False)

TautArg :: Nat -> #
TautArg (Zero) = Bool
TautArg (Succ m) = Bool -> TautArg m
\end{alltt}

\newpage
\foilhead{{\tt printf}}
\begin{alltt}
PrintfType :: String -> #
PrintfType ""           = String
PrintfType ('%':'d':cs) = Int    -> PrintfType cs
PrintfType ('%':'s':cs) = String -> PrintfType cs
PrintfType ('%': _ :cs) =           PrintfType cs
PrintfType ( _ :cs)     =           PrintfType cs

printf :: (fmt::String) -> PrintfType fmt
printf fmt = pr fmt ""

pr :: (fmt::String) -> String -> PrintfType fmt
pr ""           res = res
pr ('%':'d':cs) res = 
      \bs (i::Int)    -> pr cs (res ++ show i)
pr ('%':'s':cs) res = 
      \bs (s::String) -> pr cs (res ++ s)
pr ('%': c :cs) res =
      pr cs (res ++ [c])
pr (c:cs)       res =
      pr cs (res ++ [c])
\end{alltt}

\newpage
\foilhead{A different list representation}
\begin{alltt}
private 
ifT :: Bool -> # -> # -> #
ifT (False) t f = f
ifT (True)  t f = t

abstract
type List a = sig
    null :: Bool 
    head :: ifT null Unit a
    tail :: ifT null Unit (List a)

cons :: (a :: #) |-> a -> List a -> List a
cons x xs = struct { null = False; head = x; tail = xs }

nil :: (a :: #) |-> List a
nil |a = struct { null = True; head = unit; tail = unit }

length :: (a :: #) |-> List a -> Int
length |a l = length' |a l.null l.tail

private
length' :: (a :: #) |-> (n::Bool) -> ifT n Unit (List a) -> Int
length' |a (True)  t = 0
length' |a (False) t = 1 + length |a t
\end{alltt}

\newpage
\foilhead{Predicate logic as Cayenne types}
Use the Curry-Howard isomorphism.

\begin{tabular}{l|l}
Predicate calculus              & Cayenne type          \\
\hline 
$F$                          & \te{Absurd} (or any empty type)               \\
$T$                          & any non-empty type    \\
$x \vee y$                      & \te{Either} $x\; y$   \\
$x \wedge y$                    & \te{Pair} $x\; y$     \\
$x \supset y$                   & $x$\te{~->~}$y$     \\
$\neg x$                        & $x$\te{~->~Absurd}     \\
$\forall x \in A . P(x)$        & \te{(}$x$\te{::}$A$\te{)~->~}$P(x)$ \\
$\exists x \in A . P(x)$        & \te{\{}$x$\te{::}$A$\te{;~}$y$\te{::}$P(x)$\te{\}} \\
\end{tabular}

\begin{verbatim}

data Absurd =
data Pair x y = pair x y
data Either x y = Left x | Right y

\end{verbatim}

False propositions will correspond to empty types
and (constructively) true propositions will correspond
to non-empty types.

Finding an element in a type is the same as finding
a proof.

{\large CAVEAT} We ignore $\bot$ here; it inhabits
every type since Cayenne has general recursion.

\newpage
${\it Refl}(R) = \forall x . x R x$ \\
${\it Symm}(R) = \forall x . \forall y . x R y \supset y R x$ \\
${\it Trans}(R) = \forall x . \forall y . \forall z . x R y \wedge y R z \supset x R z$ \\

\begin{alltt}
Rel :: (a :: #) -> #1
Rel a = a -> a -> #

Refl :: (a :: #) |-> Rel a -> #
Refl |a R = (x::a) -> x `R` x

Symm :: (a :: #) |-> Rel a -> #
Symm |a R = (x::a) -> (y::a) -> x `R` y -> y `R` x

Trans :: (a :: #) |-> Rel a -> #
Trans |a R = (x::a) -> (y::a) -> (z::a) -> 
    x `R` y -> y `R` z -> x `R` z

Equiv :: (a :: #) |-> Rel a -> #
Equiv R = sig
    refl :: Refl R
    symm :: Symm R
    trans :: Trans R
\end{alltt}

\newpage
\begin{alltt}
data Absurd =

data Truth = truth

Lift :: Bool -> #
Lift (False) = Absurd
Lift (True)  = Truth

LiftBin :: (a :: #) |-> (a -> a -> Bool) -> Rel a
LiftBin |a op = \ (x::a) -> \ (y::a) -> Lift (op x y)

type Eq a = sig
    (==) :: a -> a -> Bool
    equiv :: Equiv (LiftBin (==))

\end{alltt}

With classes
\begin{alltt}
class Eq a where
    (==) :: a -> a -> Bool
    equiv :: Equiv (LiftBin (==))
\end{alltt}

\newpage
An {\tt Eq} instance for {\tt Unit}.

\begin{alltt}
data Unit = unit

Eq_Unit :: Eq Unit
Eq_Unit = struct 
    (==) (unit) (unit) = True

    equiv = struct
        refl (unit) = truth
        symm (unit) (unit) p = p
        trans (unit) (unit) (unit) p q = p
\end{alltt}

\newpage
An {\tt Eq} instance for {\tt Bool}.

\begin{alltt}
absurd :: (a :: #) |-> Absurd -> a
absurd i = case i of \{ \}

Eq_Bool :: Eq Bool
Eq_Bool = struct
    (==) (False) (False) = True
    (==) (True)  (True)  = True
    (==) _       _       = False

    equiv = struct
        refl (False) = truth
        refl (True)  = truth
        symm (False) (False) p = p
        symm (False) (True)  p = absurd p
        symm (True)  (False) p = absurd p
        symm (True)  (True)  p = p
        trans (False) (False) (False) p q = q
        trans (False) (False) (True)  p q = absurd q
        trans (False) (True)  _       p q = absurd p
        trans (True)  (False) _       p q = absurd p
        trans (True)  (True)  (False) p q = absurd q
        trans (True)  (True)  (True)  p q = q
\end{alltt}

\newpage
\foilhead{Implementation}

There is a simple implementation (not quite complete yet).
It translated to LML and compiles the LML code with type checking off.

Example.

\begin{alltt}
module foo$bar =
  "Hello, world{\bs}n"

\end{alltt}

Compile by
\begin{alltt}
cayenne bar.cy

\end{alltt}

And link by
\begin{alltt}
cayenne foo$bar

\end{alltt}

And run
\begin{alltt}
a.out

\end{alltt}

\newcommand{\nt}[1]{{\it #1}}
\newcommand{\spp}{\mbox{~~}}

\newcommand{\ruleh}[3]{\infer [$\sf \scriptsize #1$] {#3} {#2}}
\newcommand{\seph}{ & }
\newcommand{\judge}[2]{#1 \vdash #2}
\newcommand{\judget}[2]{#1 & \vdash & #2}
\newcommand{\hastype}[2]{#1 \in #2}
\newcommand{\rsep}{\vspace*{5mm}}
%\newcommand{\rsep}{\medskip}
\newcommand{\dc}{$\te{::}$}
\newcommand{\dt}{$\te{.}$}
\newcommand{\sm}{$\te{;}$}
\newcommand{\teq}{$\te{=}$}
\newcommand{\fun}{$\te{->}$}
\newcommand{\lam}{$\bs$}
\newcommand{\eq}[2]{#1 \approx #2}
\newcommand{\hash}[1]{\;$\te{\#}$_{#1}}
\newcommand{\keyw}[1]{$\te{#1}$\;}

\newcommand{\subst}[3]{#1[#2 \mapsto #3]}

\newcommand{\eval}[2]{#1 & $\longmapsto$ & #2 \\}
\newcommand{\evalu}[2]{#1 & $\longmapsto_*$ & #2 \\}
\newcommand{\ieval}[2]{$ #1 \longmapsto #2 $}
\newcommand{\ievalu}[2]{$ #1 \longmapsto_{*} #2 $}
\newcommand{\imeval}[2]{$ #1 \longmapsto^* #2 $}
\newcommand{\imevalu}[2]{$ #1 \longmapsto_{*}^* #2 $}

\newpage
\foilhead{Typing rules}
\ruleh{Star}{\mbox{}}
            {\judge{\Gamma}
                   {\hastype{\hash{n}}
                            {\hash{n+1}}}}
\rsep
\ruleh{Var}{\judge{\Gamma}{\hastype{A}{s}}}
           {\judge{\Gamma,\hastype{x}{A}}{\hastype{x}{A}}}
\rsep
\ruleh{Pi}
    {\judge{\Gamma}{\hastype{A}{s}} \seph
     \judge{\Gamma,\hastype{x}{A}}{\hastype{B}{t}}}
    {\judge{\Gamma}{\hastype{(x\dc A)\fun B}{t}}}
\rsep
\ruleh{Lam}
    {\judge{\Gamma,\hastype{x}{A}}{\hastype{b}{B}} \seph 
     \judge{\Gamma}{\hastype{(x\dc A)\fun B}{t}}}
    {\judge{\Gamma}{\hastype{\lam(x\dc A)\fun b}{(x\dc A)\fun B}}}
\rsep
\ruleh{App}{\judge{\Gamma}{\hastype{f}{(x\dc A)\fun B}} \seph \judge{\Gamma}{\hastype{a}{A}}}
           {\judge{\Gamma}{\hastype{f\;a}{\subst{B}{x}{a}}}}
\rsep
\ruleh{Data}
    {\judge{\Gamma}{\hastype{A_1}{\hash{1}}} \seph \ldots \seph
     \judge{\Gamma}{\hastype{A_n}{\hash{1}}}}
    {\judge{\Gamma}{\hastype{\;$\te{data}$\; C_1\;A_1 $\te{|}$ \ldots$\te{|}$\; C_n\;A_n }{\hash{1}}}}
\rsep
\ruleh{Con}
    {\judge{\Gamma}{\hastype{T}{s}}}
    {\judge{\Gamma}{\hastype{C_k$\te{@}$T}{A_k \fun T}}}
where $T \equiv \;$\te{data}$\; C_1\;A_1 $\te{|}$ \ldots$\te{|}$\; C_n\;A_n$

\rsep
\ruleh{Case}
{
\begin{array}{rcl}
\judget{\Gamma}{\hastype{x}{\;$\te{data}$\; C_1\;A_1 $\te{|}$ \ldots$\te{|}$\; C_n\;A_n}} \\
\judget{\Gamma,\hastype{x_1}{A_1}}{\hastype{e_1}{\subst{A}{x}{C_1\;x_1}}} \\
&\vdots&\\
\judget{\Gamma,\hastype{x_n}{A_n}}{\hastype{e_n}{\subst{A}{x}{C_n\;x_n}}} \\
\end{array}
}
{\judge{\Gamma}
  {\hastype{$\te{case}$\;x\;$\te{of}$\; C_1\;x_1 \fun e_1 \sm \ldots \sm C_n\;x_n \fun e_n }
           {A}
  }
}
\rsep
\ruleh{Prod}
{
\begin{array}{rcl}
\judget{\Gamma}{\hastype{A_1}{\hash{u_1}}} \\
\judget{\Gamma,\;\hastype{l_1}{A_1}}{\hastype{A_2}{\hash{u_2}}} \\
& \vdots & \\
\judget{\Gamma,\;\hastype{l_1}{A_1},\cdots,\hastype{l_{n-1}}{A_{n-1}}}{\hastype{A_n}{\hash{u_n}}} \\
& \vdots & \\
\judget{\Gamma,\;\hastype{l_1}{A_1},\cdots,\hastype{l_n}{A_n}}{\hastype{e_j}{A_j}} \\
& \vdots & \\
\end{array}
}
{\judge{\Gamma}
  {\hastype
    {$\te{sig\{}$\; l_1 \dc A_1 \gamma_1 \sm \ldots l_n \dc A_n \gamma_n $\te{\}}$\;}
    {\hash{min\{u_i\}}}
  }
}
where all $\gamma_i$ are either empty or ``$= e_i$''\\

\rsep
\ruleh{Rec}
{
\begin{array}{rcl}
\judget{\Gamma}{\hastype{A_1}{s_1}} \\
\judget{\Gamma,\Delta}{\hastype{e_1}{A_1}} \\
\judget{\Gamma,\;\hastype{l_1}{A_1}}{\hastype{A_2}{s_2}} \\
\judget{\Gamma,\Delta}{\hastype{e_2}{A_2}} \\
& \vdots & \\
\judget{\Gamma,\;\hastype{l_1}{A_1},\cdots,\hastype{l_{n-1}}{A_{n-1}}}{\hastype{A_n}{s_n}} \\
\judget{\Gamma,\Delta}{\hastype{e_n}{A_n}}
\end{array}
}
    {\judge{\Gamma}{
\begin{array}{c}
$\te{struct\{}$\; p_1\, a_1\, l_1 \dc A_1 \teq e_1 \sm \ldots  p_n\,a_n\,l_n \dc A_n \teq e_n $\te{\}}$\\
\hastype{\mbox{}}{\;$\te{sig\{}$\; \ldots l_i \dc A_i \gamma_i \sm \ldots $\te{\}}$\;}
\end{array}
    }}
where $l_i$ is present iff $p_i =\;$\te{public},\\
$\gamma_i$ is ``$= e_i$'' if $a_i =\;$\te{concrete} otherwise empty\\
$\Delta \equiv \hastype{l_1}{A_1 = e_1},\cdots,\hastype{l_n}{A_n = e_n}$

\rsep
\ruleh{Sel}
    {\judge{\Gamma}{\hastype{e}{\;$\te{sig\{}$\; \ldots l_i \dc A_i \gamma_i \sm \ldots $\te{\}}$\;}}}
    {\judge{\Gamma}{\hastype{e \dt l_i}{\subst{A_i}{\ldots,l_k}{e\dt l_k,\ldots} }}}

\rsep

\ruleh{Conv}
    {\judge{\Gamma}{\hastype{a}{A}} \seph
     \judge{\Gamma}{\hastype{B}{s}} \seph
     \judge{\Gamma}{\eq{A}{B}}}
    {\judge{\Gamma}{\hastype{a}{B}}}
\rsep
\ruleh{Weak}
    {\judge{\Gamma}{\hastype{A}{s}} \seph
     \judge{\Gamma}{\delta}}
    {\judge{\Gamma,\hastype{x}{A}}{\delta}}
\rsep
\ruleh{WeakE}
    {\judge{\Gamma}{\hastype{a}{A}} \seph
     \judge{\Gamma,\hastype{x}{A}}{\delta}}
    {\judge{\Gamma,\hastype{x}{A = a}}{\delta}}
\rsep

\newpage
\foilhead{Type erasure}

\begin{tabular}{rcl}
$($\bs$(x\dc t)\fun f)^*$ & $\rightarrow$ & \bs$x\fun f^*$, if $t\in\;$\te{\#} \\
$($\bs$(x\dc t)\fun f)^*$ & $\rightarrow$ & $f^*$, if $t\notin\;$\te{\#}  \\
$(f\;e)^*$ & $\rightarrow$ & $f^*\;e^*$, if $e\in t$ and $t\in\;$\te{\#} \\
$(f\;e)^*$ & $\rightarrow$ & $f^*$, if $e\in t$ and $t\notin\;$\te{\#} \\
$((x\dc t)\fun f)^*$ & $\rightarrow$ & $\perp$ \\

\te{struct\{}$\;l_1\dc t_1\teq e_1\sm \ldots\;l_n\dc t_n\teq e_n$\te{\}}$^*$ & $\rightarrow$ 
        & \te{struct\{}$\ldots\;l_k\teq e_k^*\ldots$\te{\}}, where $l_k$ is kept iff $t_k\in\;$\te{\#} \\
$(e\dt l)^*$ & $\rightarrow$ & $e^*\dt l$ \\
\te{sig\{} $\;\ldots\;$ \te{\}} & $\rightarrow$ & $\perp$ \\

$(C$\te{@}$t)^*$ & $\rightarrow$ & $C$ \\
$($\te{case}$\;e\;$\te{of}$\;C_1\;x_1\fun \;e_1\sm \ldots\;C_n\;x_n\fun \;e_n)^*$ & $\rightarrow$ 
        & \te{case}$\;e^*\;$\te{of}$\;C_1\fun \;e_1^*\sm \ldots\;C_n\fun \;e_n^*$ \\
\te{data}$\;\ldots\;^*$ & $\rightarrow$ & $\perp$ \\

\te{\#}$^*$ & $\rightarrow$ & $\perp$ \\
$x^*$ & $\rightarrow$ & $x$, if $x\in t, t\in\;$\te{\#} \\
$x^*$ & $\rightarrow$ & $\perp$, if $x\in t, t\notin\;$\te{\#} \\

\end{tabular}

\newpage
\foilhead{Evaluation rules}

\begin{tabular}{rcl}
\eval{\te{(}\bs$x\dc t\fun f$\te{)}$e$} {$\subst{f}{x}{e}$}
\eval{$e\dt l_k$} {$\subst{e_k}{\ldots,l_k}{e\dt l_k,\ldots}$}
\multicolumn{3}{c}{where $e \equiv$ \te{struct\{}$\ldots\;$\te{public}$\;a_k\;l_k\dc t_k\teq e_k\sm \ldots $\te{\}}} \\
\eval{\te{case}$\;C_k$\te{@}$t\;e\;$\te{of}$\;\ldots\;C_k\;x_k\fun\;e_k\sm\ldots$} { $\subst{e_k}{x_k}{e}$ }
\end{tabular}

\vspace*{10mm}

\begin{tabular}{rcl}
\evalu{\te{(}\bs$x\fun f$\te{)}$e$} {$\subst{f}{x}{e}$}
\evalu{$e\dt l_k$} {$\subst{e_k}{\ldots,l_k}{e\dt l_k,\ldots}$}
\multicolumn{3}{c}{where $e \equiv$ \te{struct\{}$\ldots\;a_k\;l_k\teq e_k\sm \ldots $\te{\}}} \\
\evalu{\te{case}$\;C_k\;e\;$\te{of}$\;\ldots\;C_k\;x_k\fun\;e_k\sm\ldots$} 
  { $\subst{e_k}{x_k}{e}$ }
\end{tabular}


\end{document}
