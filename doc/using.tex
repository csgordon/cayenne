%\batchmode
\nonstopmode

%%% \documentclass{acmconf}     %% ICFP
\documentclass{llncs}[12pt]             %% Portugal

\usepackage[latin1]{inputenc}
\usepackage[T1]{fontenc}

\usepackage{proof}
\pagestyle{empty}
\thispagestyle{empty}
%\renewcommand{\thepage}{}

\newcommand\bs{\char '134}
%\newcommand\notyet[1]{#1}
\newcommand\notyet[1]{}
\newcommand\com[1]{}
\newcommand\nocom[1]{#1}
\newcommand\exer{$\bullet$}

%\leftmargin -2cm
%\leftmargini 0cm
%\leftmarginv 0cm
%\leftmarginvi 0cm
%\textwidth 16cm
%\topmargin -1.0cm
%\textheight 24.5cm
%\setlength\leftmargin {-2cm}

\newcommand{\nt}[1]{{\it #1}}
\newcommand{\te}[1]{{\tt #1}}
\newcommand{\spp}{\mbox{~~}}

\newcommand{\ruleh}[3]{\infer [$\sf \scriptsize #1$] {#3} {#2}}
\newcommand{\seph}{ & }
\newcommand{\judge}[2]{#1 \vdash #2}
\newcommand{\judget}[2]{#1 & \vdash & #2}
\newcommand{\hastype}[2]{#1 \in #2}
%\newcommand{\rsep}{\vspace*{5mm}}
\newcommand{\rsep}{\medskip\medskip}
\newcommand{\dc}{$\te{::}$}
\newcommand{\dt}{$\te{.}$}
\newcommand{\sm}{$\te{;}$}
\newcommand{\teq}{$\te{=}$}
\newcommand{\fun}{$\te{->}$}
\newcommand{\lam}{$\bs$}
\newcommand{\eq}[2]{#1 \approx #2}
\newcommand{\hash}[1]{\;$\te{\#}$_{#1}}
\newcommand{\keyw}[1]{$\te{#1}$\;}

\newcommand{\subst}[3]{#1[#2 \mapsto #3]}

\def\mapstou{\mapstochar\rightharpoonup}
\newcommand{\eval}[2]{#1 & $\longmapsto$ & #2 \\}
\newcommand{\evalu}[2]{#1 & $\mapstou$ & #2 \\}
\newcommand{\ieval}[2]{$ #1 \longmapsto #2 $}
\newcommand{\ievalu}[2]{$ #1 \mapstou #2 $}
\newcommand{\imeval}[2]{$ #1 \longmapsto^* #2 $}
\newcommand{\imevalu}[2]{$ #1 \mapstou^* #2 $}

\newcommand{\begindescrlist}[1]{
\begin{list}{\arabic{enumi}}{
                \settowidth{\labelwidth}{#1}
                \setlength{\leftmargin}{\labelwidth} % {#1}
                \addtolength{\leftmargin}{\labelsep}
                \setlength{\parsep}{0ex}
                \setlength{\itemsep}{0ex}
                \usecounter{enumi}
        }
}
\newcommand{\litem}[1]{\item[#1\hfill]}

\author{
Lennart Augustsson \\
}
\institute{ }
\date{}
\title{
Using Cayenne
}


\begin{document}
\maketitle

\section{Compiling and running Cayenne programs}

Cayenne has a traditional batch compiler, and --- unfortunately ---
no interactive system yet.  Program development follows the old
edit-compile-link-run tradition.

The Cayenne compiler is still very experimental, so expect bugs and
bad error messages.


\subsection{Compiling a file}
A file with the suffix \te{.cy} can contain one or more Cayenne modules.
The file can be compiled with the command
\begin{verbatim}
   cayenne file.cy
\end{verbatim}
This will compile the modules and place the generated interface 
(which is a readable text file) and
object files in the destination directory.  The destination is
determined by the module name(s) in the file.  A module named
\te{foo\$bar} will have its interface file placed in
\te{./foo/bar.ci}.
The top destination can be set by the \te{-d} flag.

\subsection{Linking}
To produce an executable program a module with the main function has
to be named.  The link command is
\begin{verbatim}
   cayenne module$name
\end{verbatim}
The named module should be of type 
\te{System\$IO.IO~System\$Unit.Unit}.

Note that the \te{\$} needs to be quoted in most shells.

The Cayenne compiler will automatically locate the needed modules
and link them together.

\subsection{Compiler flags}
There are a number of flags to direct the Cayenne compiler to
the right location for object and interface files.

\begindescrlist{xxxxxxxxx}
\litem{\te{-help}}                 generate help message
\litem{\te{-i dir}}                directory for Cayenne system files
\litem{\te{-d dir}}                destination for generated files
\litem{\te{-o file}}               name of executable
\litem{\te{-p path}}               search path for interface and object files
\litem{\te{-v}}                    same as -fverbose
\end{list}

\subsubsection{Less useful flags}
These flags are only useful if you want to examine the internal
machinery of the Cayenne compiler.
\begindescrlist{xxxxxxxxx}
\litem{\te{-eval}}                 evaluate and print result
\litem{\te{-fverbose}}             be talkative
\litem{\te{-lml}}                  generate LML
\litem{\te{-lmlComp}}              compile LML
\litem{\te{-m}}                    keep LML
\litem{\te{-tcheck}}               type check program
\end{list}

\subsection{Example}
With the file \te{hello.cy} containing
\begin{verbatim}
module foo$hello =
#include Prelude
putStr ("Hello, world\n")
\end{verbatim}
Compile with
\begin{verbatim}
  cayenne hello.cy
\end{verbatim}
and link with
\begin{verbatim}
  cayenne -o hello foo$hello
\end{verbatim}
and finally run with
\begin{verbatim}
  ./hello
\end{verbatim}

\section{System modules}
Cayenne comes with a number of predefined modules.  These modules
have no special status, except that some syntactic constructs (like numerals
and strings) involve a type that is defined in one of these modules.

You can find some module signature in appendix \ref{app:sysmod}.
For the full list of system modules and their interfaces, please consult the
online documentation.

\subsection{I/O}
Cayenne supports monadic I/O just like Haskell, but the repertoire of
available operations is smaller.

The Cayenne ``\te{do}'' notation is also a little different from Haskell's.
The monad to use must be given explicitely after the \te{do} keyword.
The types of the variables bound with \te{do} must also be given
explicitely.
Example:
\begin{verbatim}
module example$io =
#include Prelude
    do Monad_IO
        (args :: List String) <- getArgs
        putStr "arg 1 is '"
        putStr (head args)
        putStr "'\n"
\end{verbatim}
\subsection{\te{\#include}}
Many of the system modules can be brought into scope by simply
putting
\begin{verbatim}
#include Prelude
\end{verbatim}
at the beginning of a module.
This expands to a number of \te{open} expressions.  The exact
expansion can be found in appendix \ref{app:inc}.

\section{Exercises}
Each subtask within an exercise is marked with a bullet, \exer.

\subsection{Getting started}
\exer Make sure you know the basics of Cayenne by writing a program
that prints the 20th Fibonacci number.  \exer Also try to use the list monad to
generate a list of Fibonacci numbers to print.

\subsection{Parametric modules}
\exer Define an abstract data type (a \te{struct}) for sets.  Assume that
the elements of the set only admits equality.

There are (at least) two ways of defining such an abstract data type.
The way you would do it in Haskell would get a signature like this
in Cayenne:
\begin{verbatim}
    sig
        Set :: (a :: #) -> #
        union :: (a :: #) |-> (a -> a -> Bool) -> Set a -> Set a -> Set a
        ...
\end{verbatim}
Each operation on the set takes the equality operation as an argument.
(In Haskell the class system will take care of passing this argument
around.)  In Cayenne you can also define it in a different way, where the
type and equality parameters are for the whole data type instead
of for each operation.  Then we would get:
\begin{verbatim}
    (a :: #) -> (a -> a -> Bool) -> 
    sig
        Set :: #
        union :: Set -> Set -> Set
        ...
\end{verbatim}

\exer Why can't this be done in Haskell?  (It can be done in SML.)
What are the advantages and disadvantages of these approaches?

\exer Whichever version you chose to implement, make sure you do the other one
as well.  \exer Use both of them for some examples of your own choice.

\exer Write a function (which would be a functor in SML) that takes 
an abstract data type which is a set and returns an abstract
data type which is a graph.  For this exercise use the mathematical
definition of a graph:  A graph is a set of nodes and a set of edges,
where an edge is a pair of nodes.

\subsection{A simple dependent function type}
\exer Write a function \te{neg} that takes two arguments.  The first
argument should be of type \te{BorI}:
\begin{verbatim}
  data BorI = B | I
\end{verbatim}
If the first argument is \te{B} the second argument should be of
type \te{Bool} and the function should perform logical negation.
If the first argument is \te{I} the second argument should be of
type \te{Int} and the function should perform arithmetic negation.

\exer Write a program that tests the \te{neg} function.

\subsection{A useful(?) function with a dependent type}
We would like a function that behaves as follows (using Haskell notation):
\begin{verbatim}
  apply f [] = f
  apply f [x1] = f x1
  apply f [x1,x2] = f x1 x2
  apply f [x1,x2,x3] = f x1 x2 x3
  ...
\end{verbatim}

\exer Figure out what type this function has and then implement it
in Cayenne.  Hint: it's easier if you swap the arguments to \te{apply}.

\subsection{Dependent records}
In languages like Pascal, tree-like data types are often represented
by a record with variants.  The record has a tag and depending on
the value of the tag the record contains different things.
In functional languages this is solved in a different way by
using data types where the union is always at the top level.
The purpose of this excercise is to show how the Pascal way
of doing things can be mimicked in Cayenne.

\com{
Define a type \te{BinTree} of binary trees of that have
\te{Int}s in the leaves and no information in the interior nodes.
A node in the tree should be represented by a record with two fields.
The first field is a tag that describes what kind of node it is.
The possible values of this field should be \te{Leaf}, \te{One}, and \te{Two}.
The second field contains the data.  Its type will depend on the tag.
It will contain an \te{Int}, one, or two branches.

Define a function that sums all the leaves of such a tree.
}

(Excuse the contrived example.)
\exer Define a record type \te{Vehicle} which has a field \te{brand} which
is a \te{String}, field \te{sort} of type 
``\te{data Bicycle | Car | Train}'', and a field \te{info}.  The \te{info}
field should have type \te{Bool} if it is a bicycle (to indicate if it is
a tandem), type \te{Int} if it is a car (the weight), and type \te{String}
(name of the train) if it is a train.

\exer Define lists using this technique.  A list should be a record with a
field that indicates if the list is empty or not.  There should also
be a head and a tail field.  If the list is empty these field should
be of type \te{Unit} otherwise of the appropriate type.  \exer Also define the
length function on these lists.

\subsection{Generic programming}
Consider the following data type
\begin{verbatim}
  data Type = 
    TProd Type Type | TSum Type Type | TList Type | TInt | TBool
\end{verbatim}
This data type can be used to encode the structure of (some) Cayenne
types.

\exer Write a function ``\te{decode~::~Type~->~\#}'' that takes a value of
this type and returns the corresponding type.

\exer Write a function \te{sum} that takes a value of type \te{Type} and
the a value of the of the type encoded by the first argument.
It should return the sum of all \te{Int} components of the second
argument.

\exer Try extending \te{Type} so it can be used to encode data types with
one argument (possibly recursive).

A little module excerise: \exer Try putting the definition of \te{Type}
and \te{decode} in one module and \te{sum} in another.  What
do you need to export from the first one to make it work?  Why?

\subsection{Specifications and proofs}
A Cayenne function that inserts an element into a sorted list
can be defined like this:
\begin{verbatim}
insert :: (a :: #) |-> (a -> a -> Bool) -> a -> List a -> List a
insert (<=) x (Nil) = x : Nil
insert (<=) x (x' : xs) = if (x <= x')
                          (x : x' : xs)
                          (x' : insert (<=) x xs)
\end{verbatim}
\exer What property (properties) do you want such an insert function to
have?  \exer Formulate that property in predicate calculus.  \exer Formulate
that property in Cayenne (as a type).  \exer Within Cayenne, prove that
\te{insert} has that property.  (The last part is rather tricky
and tedious if you have no previous experience of this kind of
proof system.)
%%\vspace{2cm}

%%\pagebreak
\appendix
\section{Some System modules}
\label{app:sysmod}

All the signatures below should be read as if in the context
\begin{verbatim}
let String = System$String.String
    Char = System$Char.Char
    Int = System$Int.Int
    Bool = System$Bool.Bool
    List = System$List.List
    Unit = System$Unit.Unit
    Pair = System$Tuples.Pair
in
\end{verbatim}

\subsection{Bool}
\begin{verbatim}
System$Bool ::
sig {
  data Bool = False | True;
  if :: (a :: #) |-> Bool -> a -> a -> a;
  (&&) :: Bool -> Bool -> Bool;
  (||) :: Bool -> Bool -> Bool;
  not :: Bool -> Bool;
  show :: Bool -> String;
};
\end{verbatim}

\subsection{Char}
\begin{verbatim}
System$Char ::
sig {
  type Char = System$CharType.Char;
  ord :: Char -> Int;
  chr :: Int -> Char;
  (==) :: Char -> Char -> Bool;
  (/=) :: Char -> Char -> Bool;
  (<=) :: Char -> Char -> Bool;
  (>=) :: Char -> Char -> Bool;
  (<) :: Char -> Char -> Bool;
  (>) :: Char -> Char -> Bool;
  show :: Char -> String;
};
\end{verbatim}

\subsection{Error}
\begin{verbatim}
System$Error ::
sig {
  error :: (a :: #) |-> String -> a;
  undefined :: (a :: #) |-> a;
  type Undefined;
  UndefinedT :: #1;
};
\end{verbatim}

\subsection{HO}
\begin{verbatim}
System$HO ::
sig {
  const :: (a :: #) |-> (b :: #) |-> a -> b -> a;
  id :: (a :: #) |-> a -> a;
  (�) ::
    (a :: #) |-> (b :: #) |-> (c :: #) |-> 
    (b -> c) -> (a -> b) -> a -> c;
  flip ::
    (a :: #) |-> (b :: #) |-> (c :: #) |-> 
    (a -> b -> c) -> b -> a -> c;
  curry ::
    (a :: #) |-> (b :: #) |-> (c :: #) |-> 
    (Pair a b -> c) -> a -> b -> c;
  uncurry ::
    (a :: #) |-> (b :: #) |-> (c :: #) |-> 
    (a -> b -> c) -> Pair a b -> c;
};
\end{verbatim}

\subsection{Int}
\begin{verbatim}
System$Int ::
sig {
  type Int;
  (+) :: Int -> Int -> Int;
  (-) :: Int -> Int -> Int;
  (*) :: Int -> Int -> Int;
  quot :: Int -> Int -> Int;
  rem :: Int -> Int -> Int;
  negate :: Int -> Int;
  odd :: Int -> Bool;
  even :: Int -> Bool;
  (==) :: Int -> Int -> Bool;
  (/=) :: Int -> Int -> Bool;
  (<=) :: Int -> Int -> Bool;
  (>=) :: Int -> Int -> Bool;
  (<) :: Int -> Int -> Bool;
  (>) :: Int -> Int -> Bool;
  show :: Int -> String;
};
\end{verbatim}

\subsection{List}
\begin{verbatim}
System$List ::
sig {
  data List a = Nil | (:) a (List a);
  null :: (a :: #) |-> List a -> Bool;
  length :: (a :: #) |-> List a -> Int;
  map :: (a :: #) |-> (b :: #) |-> (a -> b) -> List a -> List b;
  filter :: (a :: #) |-> (a -> Bool) -> List a -> List a;
  foldr :: (a :: #) |-> (b :: #) |-> (a -> b -> b) -> b -> List a -> b;
  concat :: (a :: #) |-> List (List a) -> List a;
  head :: (a :: #) |-> List a -> a;
  tail :: (a :: #) |-> List a -> List a;
  show :: (a :: #) |-> (a -> String) -> List a -> String;
  (++) :: (a :: #) |-> List a -> List a -> List a;
  take :: (a :: #) |-> Int -> List a -> List a;
  drop :: (a :: #) |-> Int -> List a -> List a;
  zipWith ::
    (a :: #) |-> (b :: #) |-> (c :: #) |-> 
    (a -> b -> c) -> List a -> List b -> List c;
  Monad_List :: System$Monad List;
};
\end{verbatim}

\subsection{Maybe}
\begin{verbatim}
System$Maybe ::
sig {
  data Maybe a = Nothing | Just a;
  maybe :: (a :: #) |-> (b :: #) |-> b -> (a -> b) -> Maybe a -> b;
  fromJust :: (a :: #) |-> Maybe a -> a;
  Monad_Maybe :: System$Monad Maybe;
  show :: (a :: #) |-> (a -> String) -> Maybe a -> String;
};
\end{verbatim}

\subsection{Monad}
\begin{verbatim}
type System$Monad (m :: # -> #) = 
sig {
  (>>=) :: (a :: #) |-> (b :: #) |-> m a -> (a -> m b) -> m b;
  (>>) :: (a :: #) |-> (b :: #) |-> m a -> m b -> m b;
  return :: (a :: #) |-> a -> m a;
};
\end{verbatim}

\subsection{Tuples}
\begin{verbatim}
System$Tuples ::
sig {
  data Pair a b = (,) a b;
  fst :: (a :: #) |-> (b :: #) |-> Pair a b -> a;
  snd :: (a :: #) |-> (b :: #) |-> Pair a b -> b;
  show :: (a :: #) |-> (b :: #) |->
          (a -> String) -> (b -> String) -> Pair a b -> String;
};
\end{verbatim}

\subsection{Unit}
\begin{verbatim}
System$Unit ::
sig {
  data Unit = unit;
  show :: Unit -> String;
};
\end{verbatim}

\section{\te{\#include Prelude}}
\label{app:inc}
This is text inserted
\begin{verbatim}
open System$Unit   use * in
open System$String use * in
open System$List   use * in
open System$Bool   use * in
open System$Char   use * in
open System$Error  use * in
open System$IO     use * in
open System$Tuples use * in
open System$Either use * in
open System$HO     use * in
open System$Int    use * in
\end{verbatim}

\end{document}

