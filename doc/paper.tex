%\batchmode
\nonstopmode

%%% \documentclass{acmconf}     %% ICFP
\documentclass{llncs}[12pt]             %% Portugal

\usepackage[latin1]{inputenc}
\usepackage[T1]{fontenc}

\usepackage{proof}
\pagestyle{empty}
\thispagestyle{empty}
%\renewcommand{\thepage}{}

\newcommand\bs{\char '134}
%\newcommand\notyet[1]{#1}
\newcommand\notyet[1]{}
\newcommand\com[1]{}
\newcommand\nocom[1]{#1}

%\leftmargin -2cm
%\leftmargini 0cm
%\leftmarginv 0cm
%\leftmarginvi 0cm
%\textwidth 16cm
%\topmargin -1.0cm
%\textheight 24.5cm
%\setlength\leftmargin {-2cm}

\newcommand{\nt}[1]{{\it #1}}
\newcommand{\te}[1]{{\tt #1}}
\newcommand{\spp}{\mbox{~~}}

\newcommand{\ruleh}[3]{\infer [$\sf \scriptsize #1$] {#3} {#2}}
\newcommand{\seph}{ & }
\newcommand{\judge}[2]{#1 \vdash #2}
\newcommand{\judget}[2]{#1 & \vdash & #2}
\newcommand{\hastype}[2]{#1 \in #2}
%\newcommand{\rsep}{\vspace*{5mm}}
\newcommand{\rsep}{\medskip\medskip}
\newcommand{\dc}{$\te{::}$}
\newcommand{\dt}{$\te{.}$}
\newcommand{\sm}{$\te{;}$}
\newcommand{\teq}{$\te{=}$}
\newcommand{\fun}{$\te{->}$}
%\newcommand{\lam}{$\bs$}
\newcommand{\lam}{$\te{\bs}$}
\newcommand{\eq}[2]{#1 \approx #2}
\newcommand{\hash}[1]{\;$\te{\#}$_{#1}}
\newcommand{\keyw}[1]{$\te{#1}$\;}

\newcommand{\subst}[3]{#1[#2 \mapsto #3]}

\def\mapstou{\mapstochar\rightharpoonup}
\newcommand{\eval}[2]{#1 & $\longmapsto$ & #2 \\}
\newcommand{\evalu}[2]{#1 & $\mapstou$ & #2 \\}
\newcommand{\ieval}[2]{$ #1 \longmapsto #2 $}
\newcommand{\ievalu}[2]{$ #1 \mapstou #2 $}
\newcommand{\imeval}[2]{$ #1 \longmapsto^* #2 $}
\newcommand{\imevalu}[2]{$ #1 \mapstou^* #2 $}


\author{
Lennart Augustsson \\
}
%%\affiliation{
\institute{
{ Department of Computing Sciences}\\
{ Chalmers University of Technology}\\
{ S-412 96 G{\"o}teborg, Sweden}\\
Email: {\tt augustss@cs.chalmers.se}\\
WWW: {\tt http://www.cs.chalmers.se/\~{}augustss}\\
%% Phone: + 46 31 772 10 42\\
%% Fax: + 46 31 16 56 55\\
}
\date{}
\title{
Cayenne ---
a language with dependent types
}

\begin{document}
\maketitle

\begin{abstract}
Cayenne is a Haskell-like language.  The main difference between Haskell and Cayenne
is that Cayenne has dependent types, i.e., the {\em result type} of a function
may depend on the {\em argument value}, and types of record components 
(which can be types or values)
may depend on other components.  Cayenne also combines the
syntactic categories for value expressions and type expressions; thus
reducing the number of language concepts.

Having dependent types and combined type and value expressions makes the
language very powerful.  It is powerful enough that a special module
concept is unnecessary; ordinary records suffice.  It is also powerful
enough to encode predicate logic at the type level, allowing types to
be used as
specifications of programs.  However, this power comes at a cost:
type checking of Cayenne is undecidable.  While this may appear
to be a steep price to pay, it seems to work well in practice.

{\bf Keywords}: Type systems, language design, dependent types, module
systems

\end{abstract}

%% \copyrightspace %% ACM

\section{Introduction}
Languages like Haskell \cite{hudak:haskell} and SML
\cite{milner:sml-standard} have type systems that are among the most advanced 
of any language.  Despite this there are things that are inexpressible
in these type systems.  Dependent types, i.e., having types depend on
values, increases the expressiveness of type systems and many of the
problems of Hindley-Milner typing can be overcome.

Cayenne is a Haskell-like language that combines dependent types and
first class types, i.e., types can be used like values.  The syntax for value
and type expressions is the same.  Cayenne does not have a separate
notion of modules; records are used as modules, this means that the
language for combining modules is also the usual expression language.
This is in contrast with Haskell and SML.  Haskell has similar but
different syntax for type and value expressions and definitions.
SML has different syntax for value, type, and module expressions and definitions.
It can be argued that they should look different, because they are different.
But we want to argue the opposite, the facilities for the three types
of expressions are similar, so why should they be different?  In Cayenne
they are the same and exactly the same program constructs can be used
on all levels, thus reducing the number of concepts that you need to master.

Although dependent types have been used before in proof systems, e.g.,
\cite{coquand:information},
to our knowledge this is the first time that the full power
of dependent types has been integrated into a
{\em programming language}.

We will now give some motivating examples, where we show problems
in Haskell that are solved in Cayenne.  The differences between
Haskell and Cayenne will be explained as they occur.

\subsection{The type of {\tt printf}}
The C standard I/O library 
%\cite{stdio}
has a very useful function for
doing output, namely {\tt printf}.  The function {\tt printf} takes a formatting
string as the first argument and then some additional arguments.  The
number of arguments and their types depends on the formatting
string.  It is simple to write a similar function in Haskell, but it will
not type check.\footnote{The code given here is very inefficient, but
that is easy to remedy.}
\begin{verbatim}
printf fmt = pr fmt "" where
  pr ""           res = res
  pr ('%':'d':cs) res = 
      \ i -> pr cs (res ++ show (i::Int))
  pr ('%':'s':cs) res = 
      \ s -> pr cs (res ++ s)
  pr ('%': c :cs) res =
      pr cs (res ++ [c])
  pr (c:cs)       res =
      pr cs (res ++ [c])
\end{verbatim}
This is a very simplified version of {\tt printf}, but as in the real
version, the substring ``{\tt \%d}'' marks an integer argument and 
``{\tt \%s}'' marks a string argument.  The type of {\tt printf} clearly
varies with its first argument; e.g.,
%%\\ \\
%%{\tt \spp printf~"\%d"~::~Int~->~String}\\
%%{\tt \spp printf~"\%s~owes~\%d~SEK~to~\%s"~::\\ 
%%     \spp\spp String~->~Int~->~String~->~String}\\
\begin{verbatim}
  printf "%d" :: Int -> String
  printf "%s owes %d SEK to %s" ::
      String -> Int -> String -> String
\end{verbatim}

As we can see, the function is easy to write and works perfectly, but
cannot be given a type in Haskell.\footnote
{Olivier Danvy has recently shown, \cite{danvy:formatting},
that functions similar to {\tt printf} can be given a type
with Hindley-Milner typing with a clever trick.}

\subsubsection*{Cayenne solution}
The type of {\tt printf} can easily be computed from the first argument.
All we need to do is to write a function that computes the right type.
The type of all types is called ``{\tt \#}''\footnote{
We would like to
use the more familiar notation ``{\tt *}'' for the type of types.  
This might be possible, but it interacts badly with the use of ``{\tt
*}'' as an infix operator.}
in Cayenne.
\begin{verbatim}
PrintfType :: String -> #
PrintfType ""           = String
PrintfType ('%':'d':cs) = Int    -> PrintfType cs
PrintfType ('%':'s':cs) = String -> PrintfType cs
PrintfType ('%': _ :cs) =           PrintfType cs
PrintfType ( _ :cs)     =           PrintfType cs

printf :: (fmt::String) -> PrintfType fmt
printf fmt = pr fmt ""

pr :: (fmt::String) -> String -> PrintfType fmt
pr ""           res = res
pr ('%':'d':cs) res = 
      \ (i::Int)    -> pr cs (res ++ show i)
pr ('%':'s':cs) res = 
      \ (s::String) -> pr cs (res ++ s)
pr ('%': c :cs) res =
      pr cs (res ++ [c])
pr (c:cs)       res =
      pr cs (res ++ [c])
\end{verbatim}
The function {\tt PrintfType} mimics the recursive structure of {\tt
printf}, but it computes the type instead of the value.  E.g.,\\
{\tt \spp PrintfType~"\%d"~$\longmapsto$~Int~->~String}\\
The typing of {\tt printf} is now\\
{\tt \spp printf~::~(fmt::String)~->~PrintfType~fmt}\\
This example differs from Haskell in that the first argument
(which has type {\tt String}) has a name, {\tt fmt}, which can be used
in the type expression.  A minor point to note is that
$\lambda$-expressions in Cayenne have an explicit type on the bound
variable, whereas they do not in Haskell.

Another example of a function with a dependent type can be found
in appendix \ref{app:taut}.

\subsection{The set ``package''} 
Record types in Haskell (and SML) can contain values, but not types;
sometimes this can be inconvenient. To show an example of this we will
use a simple set of integers.  It should
support creating the empty set, the singleton set, taking union, and
testing for set membership.  There are many possible ways to implement
these sets and sometimes you want to have multiple implementations
in a program and choose dynamically which one to use (e.g., depending
on the use pattern).  To be able to do this we would like to be able
to store different set implementations in a data structure.

We would want to define the set type something like
\begin{verbatim}

data IntSet = IntSet { 
    type T,
    empty :: T,
    singleton :: Int->T,
    union :: T->T->T,
    member :: Int->T->Bool
    }
\end{verbatim}

Unfortunately, this is not possible since we cannot have a type in a record and
the name {\tt T} would also not be in scope.
This kind of construct is only available at the module level in
Haskell, but modules are definitely not first class objects in
Haskell; there are no operations on modules except for the
importation of them.
SML allows this kind of definitions on the module level and has a rich
language for combining them, but they are still not first class
objects, so they cannot be put in a run-time data structure.\footnote{It is not obvious that the first class modules proposed
here extend easily to a language like SML that supports side
effects.}

\subsubsection*{Cayenne solution}
Cayenne records are different from Haskell records in several respects:  Cayenne records
are not data types, they can contain types, and when defining a record
object the labels are bound within the record expression.  
The {\tt sig} keyword starts a record type 
and the {\tt struct} keyword starts a record value.

The {\tt IntSet} type could be defined like this
\begin{verbatim}
type IntSet = sig
    type T
    empty :: T
    singleton :: Int->T
    union :: T->T->T
    member :: Int->T->Bool
\end{verbatim}

An implementation could look like this
\begin{verbatim}
na�veSet :: IntSet
na�veSet = struct
    abstract type T = Int->Bool
    empty x = False
    singleton x x' = x == x'
    union s t x = s x || t x
    member x s = s x
\end{verbatim}

This kind of record borrows features from Haskell modules, but they
are still first class objects.  
% In fact, Cayenne has no separate module concept; the records are used instead of modules.

\subsection{The {\tt Eq} class}
The {\tt Eq} class in Haskell has the following definition:\footnote{It also
has a definition of {\tt (/=)}, but it is of no use in this example so we
disregard it.}
\begin{verbatim}
class Eq a where
    (==) :: a -> a -> Bool
\end{verbatim}
This, quite correctly, states that {\tt (==)} takes two arguments of
the same type and returns a boolean, but surely this is not all we
expect from an equality.  We expect it to be a ``real'' equality,
i.e., we most likely want it to be an equivalence relation.\footnote{Or,
even better, a congruence relation.}  The equivalence property of equality
cannot be expressed in Haskell.  The best we can do is to have it
as a comment, and hope that each equality defined in the program is really
an equivalence relation.

\subsubsection*{Cayenne solution}
Cayenne has no type classes so the {\tt Eq} class problem must be
reformulated 
slightly.  A class definition in Haskell would correspond to a type
definition of a record in Cayenne, and instance declarations in
Haskell correspond to values of that type.  All dictionaries will thus
be passed explicity in Cayenne.

The {\tt Eq} ``class'' in Cayenne would be
\begin{verbatim}
type Eq a = sig
    (==) :: a -> a -> Bool
\end{verbatim}
To include an equivalence proof we must first have a way of expressing
logical properties.  This is, in fact, easy since Cayenne types can,
through the Curry-Howard isomorphism,
encode predicate calculus as types, see figure~\ref{fig:predcalc}.
Terms of the different types correspond to the
proof of the corresponding properties.
This is all well known from constructive
type theory \cite{nordstrom:book}, and well before that \cite{howard:form-as-types}.

\begin{figure}
\begin{tabular}{l|l}
Predicate calculus              & Cayenne type          \\
\hline 
$\bot$                          & \te{Absurd} (or any empty type)               \\
$\top$                          & any non-empty type    \\
$x \vee y$                      & \te{Either} $x\; y$   \\
$x \wedge y$                    & \te{Pair} $x\; y$     \\
$\forall x \in A . P(x)$        & \te{(}$x$\te{::}$A$\te{)~->~}$P(x)$ \\
$\exists x \in A . P(x)$        & \te{\{}$x$\te{::}$A$\te{;~}$y$\te{::}$P(x)$\te{\}} \\
\end{tabular}
\begin{verbatim}
data Absurd =
data Pair x y = pair x y
data Either x y = Left x | Right y
\end{verbatim}
\caption{``Encoding'' predicate logic as Cayenne types.}
\label{fig:predcalc}
\end{figure}

We encode the absurd proposition (i.e., falsity) by the empty type, and
all types with elements encode truth.  The dependent function type
encodes universal quantification and records encode existential quantification.
Proving a property correspond to finding an element (i.e., constructing a value)
in a type.  Since false logical statements correspond to the empty type
we cannot find any values in them, but in (constructively) true logical
statements we can.

One way of solving our problem in Cayenne is to extend the {\tt Eq} type like this:
\begin{verbatim}
type Eq a = sig
    (==) :: a -> a -> Bool
    equiv :: Equiv (LiftBin (==))
\end{verbatim}
{\tt LiftBin} is a function that maps a binary operation yielding a Boolean
into a corresponding relation.  {\tt Equiv} is a predicate on relations stating
that the relation is an equivalence relation.

The following auxiliary definitions are used above.  Further differences
between Haskell and Cayenne appear below: type variables must be
bound, but are often used as {\em hidden} arguments, introduced by the
{\tt |->} function arrow, see section~\ref{sec:hidden} for further discussion.
{\small
\begin{verbatim}
data Absurd =

data Truth = truth

Lift :: Bool -> #
Lift (False) = Absurd
Lift (True)  = Truth

LiftBin :: (a :: #) |-> (a -> a -> Bool) -> Rel a
LiftBin |a op = \ (x::a) -> \ (y::a) -> Lift (op x y)

type Rel a = a -> a -> #

Refl :: (a :: #) |-> Rel a -> #
Refl |a R = (x::a) -> x `R` x

Symm :: (a :: #) |-> Rel a -> #
Symm |a R = (x::a) -> (y::a) -> x `R` y -> y `R` x

Trans :: (a :: #) |-> Rel a -> #
Trans |a R = (x::a) -> (y::a) -> (z::a) -> 
    x `R` y -> y `R` z -> x `R` z

Equiv :: (a :: #) |-> Rel a -> #
Equiv R = sig
    refl :: Refl R
    symm :: Symm R
    trans :: Trans R
\end{verbatim}
}

Appendix \ref{app:equiv} contains the complete code for this example
with some instances.

\section{Core Cayenne}
Cayenne has three basic type forming constructs: dependent functions,
data types (sums), and dependent records (products).\footnote{The terminology is a little confusing here, what in constructive type theory is usually called dependent products is called
dependent functions in this paper and what in CTT is called
dependent sums is called dependent records here.  The latter terminology is more in the tradition
of programming languages.}
Core Cayenne is the subset
of Cayenne that has no syntactic bells and whistles, just the basic
constructs.  We will start by looking at Core Cayenne and then at the
various syntactic shorthands.  The syntax of Core Cayenne is given in
figure~\ref{fig:syntax}.  The grammar disregards certain minor
concrete syntax issues.  There is no syntactic distinction between
expressions and types in Cayenne, as is reflected in the grammar.

\begin{figure*}
\center{
\begin{tabular}{llll}
\nt{expr} & ::=
 & \te{(} \nt{varid} \te{::} \nt{type} \te{)} \te{->} \nt{expr} & function type\\
&& \te{\bs} \te{(} \nt{varid} \te{::} \nt{type} \te{)}
\te{->} \nt{expr} & $\lambda$ expression \\
&& \nt{expr} \nt{expr} & application \\
&& \te{data} \{ \nt{conid} \{ \nt{type} \} \te{|} \} & sum type\\
&& \nt{conid} \te{@} \nt{type} & constructor \\
&& \te{case} \nt{varid} \te{of} \{ \nt{arm} \} \te{::} \nt{type} & sum
scrutinization \\
&& \te{sig} \{ \nt{sign} \} & record type \\
&& \te{struct} \{ \nt{defn} \} & record formation \\
&& \nt{expr} \te{.} \nt{lblid} & record selection \\
&& \nt{id} & variable \\
&& \te{\#}$_n$ & type of types\\
\nt{arm} & ::= & \te{(} \nt{conid} \{ \nt{varid} \} \te{)~->} \nt{expr} \te{;} \\
         &     & \nt{varid} \te{->} \nt{expr} \te{;} \\
\nt{sign} & ::= & \nt{lblid} \te{::} \nt{type} \te{;} \\
          &     & \nt{lblid} \te{::} \nt{type} \te{=} \nt{expr} \te{;} \\
\nt{defn} & ::= & \nt{vis} \nt{lblid} \te{::} \nt{type} \te{=} \nt{expr} \te{;} \\
\nt{vis} & ::= & \te{private} | \te{public} \nt{abs} \\
\nt{abs} & ::= & \te{abstract} | \te{concrete} \\
\nt{type} & ::= & \nt{expr} \\
\nt{varid} & ::= & \nt{id} \\
\nt{conid} & ::= & \nt{id} \\
\nt{lblid} & ::= & \nt{id} \\
\end{tabular}
}
\caption{
%{\center
Core Cayenne abstract syntax grammar.
Metasyntax: \{ \} are used to denote repetition of an arbitrary number of items.
%}
}
\label{fig:syntax}
\end{figure*}

\subsection{Functions}
Function expressions are written as $\lambda$-expressions.  The bound
variable must be given a type.  The function type is written like the
$\lambda$-expression, but without the leading ``\te{\bs}''.  

The big difference between the Cayenne function type and the Haskell function
is that since the bound variable is available to the right of the arrow,
the result type of a function may depend on the value of the argument.

Function application is written with juxtaposition as usual.

Example: \\
{\tt \bs~(x::Int)~->~inc~x}\\
which has type\\
{\tt (x::Int)~->~Int}

\subsection{Data types}
Unlike Haskell, a data type (sum type)
does not have to be given a name; there is an expression that denotes
each data type.  E.g., ``{\tt data False | True}'' is the type of
booleans.

Constructors are written in a way that is very different from
Haskell.  
The constructor names used in a data type expression have no name restrictions
(unlike Haskell where they have to be capitalized)
and need not be unique.  
Consequently, given only the name of a constructor
it is impossible tell what type it constructs.  Therefore, constructors are
given with their types in Cayenne.  E.g., ``{\tt True@(data False | True)}'' is
one of the constructors for the boolean type, or, if
``\te{Bool}'' has been defined, it can be written ``{\tt True@Bool}''.
Constructor names are not part of the usual name space; they can only occur
in ``\te{@}''-expressions and \te{case} expressions and in the latter the
type that they construct can be deduced.

Case expressions in Core Cayenne look a little different from Haskell.
Only simple patterns are allowed and all constructor patterns have
to be parenthesised to distinguish them from variable patterns.
Apart from the scrutinized variable and the
case arms, the case expression also has a type attached.  This type expression
gives the type of the arms of the case expression.  Note that this
expression can contain the scrutinized variable so the type may depend on it.
The reason for having this type is that with dependent types it is not
in general possible to figure out the type of the case expression.

Example:\\
\te{
case l of\\
(Nil) -> True;\\
(Cons x xs) -> False;
\\ \mbox{~~~}:: Bool
}

An example with a dependent type:\\
\te{
case~l~of\\
(True)~~->~1;\\
(False)~->~"Hello";\\
\mbox{~~~}::~(case~l~of~(True)~~->~Int; \\
\mbox{~~~~~~~~~~~~~~~~~}(False)~->~String)
}

\subsection{Records}
The record type (product type) in Cayenne is the most complicated of the type formers.  The
reason for this is that records also serve the purpose of modules in
most other languages.

A record type is written as {\tt sig} followed by a signature for each
component of the record.  The signature normally gives only the type
of the component, but it can also give the {\em value} of it.  This
feature is sometimes called a translucent sum, and is described in
more detail in section~\ref{sec:tsum}.

A record is formed by the {\tt struct} keyword followed by bindings for
all the record components.  Each binding gives the type and value of
the component as well as its visibility.  The names of the record
components (the labels) are in scope within the record expression.
This means that the bindings are mutually recursive.\footnote{Though there
are restrictions on how the recursion may occur in the signatures to ensure
that the type can be viewed as a fixpoint of a $\Sigma$-type.}

The visibility for a record component determines how it will show up
in the type of the record.  A {\tt private} component does not show up
at all in the type of the record, a {\tt public abstract} component has only
its type, and a {\tt public concrete} component has both its type and
value in the type of the record.

A record component, which occurs in (i.e., which is not \te{private}) the type of the record, can be
extracted with the usual dot notation.

Examples:
{\tt \\struct\\
\mbox{~~~}private~x~::~Int~=~1\\
\mbox{~~~}public~abstract~y~::~Int~=~x+1~\\
\mbox{~~~}public~concrete~z~::~Int~=~x+2\\} 
has type 
{\tt \\
sig \\
\mbox{~~~}y~::~Int\\
\mbox{~~~}z~::~Int=3
} \\
Selection: \te{r.y + r.z}

\subsection{The type of types}
The type of types is $\hash{1}$, this type has type $\hash{2}$ which has
type $\hash{3}$ etc.  The reasons for using a stratified type
system are twofold: first, using ``\te{\#~::~\#}'' would, even in the absence of recursion,
make the Cayenne
type system unsound as a logic as it would allow Girard's paradox;
second, the unstratified type system would make it impossible during type
checking to determine if an expression corresponds to a type or a real value
and it would be impossible to remove the types at runtime, see section~\ref{sec:erasetypes}.

Note that there is no elimination construct for the \te{\#} type, i.e., no
\te{casetype} construct.  It would be possible and useful to have such a construct,
but Cayenne currently lacks it, partly because having it would make it
impossible to remove runtime type information, see \ref{sec:erasetypes}.

%{\sf The actual stratification used in Cayenne is a little more complicated, but is left 
%out of this version of the paper for brevity.}

\section{Full Cayenne}
Using Core Cayenne would be feasible, but quite tedious, just like
using the bare $\lambda$-calculus is.  Cayenne has many syntactic
constructs to make it more palatable and closer to an ordinary functional
language.

\subsection{Hidden arguments}
\label{sec:hidden}
Many functions have type arguments that seem to serve no purpose, except
to irritate the user. E.g. \te{\\
\spp if~::~(a~::~\#)~->~Bool~->~a~->~a~->~a\\}
for each use of {\tt if} the type of the two branches must be
given as the first argument.

To lessen this problem Cayenne uses a mechanism for leaving out certain
arguments at the application site.  However, the arguments still must be
given when the function is defined.  Hidden arguments introduce
a new version of the function type, the function abstraction, and the function
application.

The function arrow in both the type and abstraction notation is
written {\tt |->} for hidden arguments.  Application of a
hidden argument uses infix {\tt |}, but normally a hidden
argument does not need to be given at all.

In function definitions the hidden arguments should not be present on
the left hand side unless preceded by a {\tt |}, i.e., the left
hand side looks like an application.

Example:
\begin{verbatim}
if :: (a :: #) |-> Bool -> a -> a -> a
if (True)  x y = x
if (False) x y = y
\end{verbatim}
This ``{\tt if}'' function can the be used as ``{\tt if True 1 2}'', or more
explicitly ``{\tt if |Int True 1 2}''.

The concept of hidden arguments is a syntactic device without any deep semantic properties.
The function type for hidden arguments should not be viewed as a new
type.  It is completely compatible with the normal function type.
It only serves as a marker to aid the insertion of the hidden arguments.
This view of hidden arguments was presented in \cite{augustsson:alf} and later
used in other systems like Lego, \cite{luo:lego},
where the concept was formalized.
Similar mechanism exist e.g.,
in Quest, \cite{cardelli:quest}, and Russell, \cite{boehm:russell}.

The current implementation of hidden arguments is quite weak and cannot
always find the hidden arguments even when it seems reasonable that
it should.  It can find a hidden argument if the variable (\te{a} in
the example) occurs in a later argument type or the result type.
In the future we will probably switch to a more powerful
method that introduces metavariables (in the sense of logical 
frameworks) and tries to derive their values using more powerful
methods such as unification.

\subsection{Syntactic sugar}
This is a brief list of syntactic extensions that can be regarded
as mere ``sugar''.
\begin{itemize}
\item If the variable bound in the function type does not occur
anywhere it can be dropped and the
function type is thus written as in Haskell. E.g. ``{\tt (x::Int)->Int}''
can be written as ``{\tt Int->Int}'' instead.

\item Infix operators (with a fixed set of precedences) can be used.
The same conventions as in Haskell are used.

\item The patterns in case arms can be written in the normal Haskell
style with nested patterns etc.  The type part of case expression is
only necessary if the type of the right hand sides depend on the scrutinized
expression.

\item {\tt public} can be omitted, since it is the default.  {\tt concrete}
is the default for type definitions, and {\tt abstract} for other
definitions.

\item Function definitions can be written in the normal Haskell style
with type signatures and pattern matching. E.g.,
\begin{verbatim}
last :: (a::#) |-> List a -> a = 
    \ (a :: #) |-> \ (l::List a) ->
    case l of
    (x : (Nil)) -> x
    (x : xs) -> last xs
\end{verbatim}
can be written
\begin{verbatim}
last :: (a::#) |-> List a -> a
last (x : (Nil)) = x
last (x : xs) = last xs
\end{verbatim}

\item If a definition is preceded by the keyword {\tt type} it is
assumed to have type {\tt \#} and all its arguments have default type
{\tt \#}. E.g.\\
\te{~~P~::~\#~->~\#~=~\bs~(a~::~\#)~->~a->Bool}\\
can be written \\
\te{~~type~P~a~=~a->Bool}

\item A data type definition can be written in the same way as in
Haskell.  This corresponds to several bindings.  First one for the type
itself, then one for each constructor in the type.  E.g. the definition
{\tt \\ 
data Maybe a = Nothing | Just a \\
} correspends to the definitions 
\begin{verbatim}
Maybe :: # |-> # =
  \(a::#) |-> data Nothing | Just a 
Nothing :: (a::#) |-> Maybe a =
  \(a::#) |-> Nothing@(Maybe a) 
Just :: (a::#) |-> a -> Maybe a =
  \(a::#) |-> \(x::a) -> Just@(Maybe a) x
\end{verbatim}

\item Cayenne has a {\tt let} expression that is like the Haskell {\tt
let} expression.  This can be translated into a record expression.
%% ``\te{let {\it defs} in e}'' is equivalent to ``\te{struct \{ {\it defs}; x = e \}.x}''
%% where \te{x} is a fresh variable.


\item To make access to record components more convenient there is an
{\tt open} expression that ``opens'' a record and makes its components
available.  The {\tt open} construct explicitly names the components that should
be visible.  E.g. 
``{\tt open~movePoint~d~p~use~x,~y~in~dist~x~y}''.
The ``{\tt open}'' expression can easily be translated to a ``\te{let}''
expression.

\item A value of record type can be coerced to a value of a different
record type if the result type is the same as the original except that
it has fewer fields.  The coercion is written ``\nt{expr}~\te{::}~\nt{type}''
and translates to a let expression.

\item Type signatures can be omitted in many places.
Even if the basic rule is that all Cayenne definitions should
have a type signature it is easy to relax this rule somewhat.
With the relaxed rule Cayenne programs have about the same
number of type signatures as the corresponding Haskell program
would have and they place no big burden on the programmer.

\item $\hash{1}$ can be written as \te{\#}.

\item A Haskell-like ``\te{do}'' notation can be used for monads.

\item The Haskell layout rule is used to avoid braces and semicolons.
The keywords \te{case}, \te{do}, \te{let}, \te{sig}, 
and \te{struct} triggers it.

\end{itemize}

\subsection{Modules}
Cayenne does not really have any modules in the traditional sense, all
it has is named expressions that exist in a global name
space.  Module names are distiguished by having a ``{\tt \$}'' in their names.
The module name space can be viewed as hierarchical with
``{\tt \$}'' as the name separator (like how UNIX path names use ``\te{/}'' 
or how Java names use ``\te{.}'').
Module identifiers can be used freely in expressions without any
explicit import declaration (just as in Java).

A module definition looks like a simple definition except that it is
preceeded by the keyword {\tt module}.  The type in the definition is
not necessary and it can be left out.  A module can also have {\tt
concrete} visibility.  This plays the same role here as it does for records,
i.e., you can make the value of a module known instead of only its type.

Some sample modules:
\begin{verbatim}
module foo$bar = struct
data Nat = Zero | Succ Nat

module foo$baz =
open System$Int use Int, (+) in
struct
inc :: Int -> Int
inc x = x+1
dec :: Int -> Int
dec x = x-1
\end{verbatim}

%\subsubsection{{\tt \#include}}
%Most programs need values from the <samp>System</samp> modules.  To get easy access
%to the entities defined in a module it is normally <samp>open</samp>ed.  This means
%that most modules will start out with a number of <samp>open</samp> constructs.
%To alleviate this problem there are some predefined set of texts that
%can be included in a module.  These texts are inserted by simple
%text insertion with an <samp>#include</samp>.

Modules are the units of separate compilation.  To compile a module,
only the types of the modules it refers to need be known.

\section{The Cayenne type system}

\subsection{Translucent sums}
\label{sec:tsum}
Many Haskell modules export types in a non-abstract way, i.e., the
type is exported so that not only the name of the type, but also its
constructors are known.  E.g.
\begin{verbatim}
module Tree(Tree(..), depth) where
data Tree a = Leaf | Node (Tree a) a (Tree a)
depth :: Tree a -> Int
depth Leaf = 0
depth (Node l _ r) = 1 + (depth l `max` depth r)
\end{verbatim}
If we try to write the corresponding Cayenne record we get
\begin{verbatim}
module ex$Tree = struct 
data Tree a = Leaf | Node (Tree a) a (Tree a)
depth :: (a :: #) |-> Tree a -> Int
depth (Leaf) = 0
depth (Node l _ r) = 1 + (depth l `max` depth r)
\end{verbatim}
which has type
\begin{verbatim}
sig 
  Tree :: # -> #
  depth :: (a :: #) |-> Tree a -> Int
\end{verbatim}
This is definitely not what we had in mind, because from this
signature we can only see that {\tt Tree} is a type constructor, but we
cannot see its definition.  We could try and remedy this by saying that
to use a module, not only must its signature be known, but its actual
value as well. This way, we would have the definition of {\tt Tree} available.
But this is also not what we intended, because this would reveal the
definition of {\tt depth}, which we may not want to reveal to users
of the {\tt ex\$Tree} module.

For this reason we introduce the possibility for each
record component to specify if it should be fully known or only known with its
type.  We then write\footnote{The {\tt
abstract} and {\tt concrete} keywords are actually superfluous in this example
because the default visibility is the same as those indicated by
the keywords.}

\begin{verbatim}
module ex$Tree = struct 
concrete
data Tree a = Leaf | Node (Tree a) a (Tree a)
abstract 
depth :: (a :: #) |-> Tree -> Int
depth (Leaf) = 0
depth (Node l _ r) = 1 + (depth l `max` depth r)
\end{verbatim}
which has type
\begin{verbatim}
sig 
  Tree :: # -> # =
     data Leaf | Node (Tree a) a (Tree a)
  Leaf :: (a::#) |-> Tree a = 
     \ (a::#) |-> Leaf@(Tree a)
  Node :: (a::#) |-> Tree a -> a -> Tree a 
          -> Tree a =
     \ (a::#) |-> \ (l::Tree a) -> \ (x::a) -> 
     \ (r::Tree a) -> Node@(Tree a) l x y
  depth :: (a :: #) |-> Tree a -> Int
\end{verbatim}

This is a very peculiar type because it not only specifies the
types of the \te{Tree}, \te{Leaf}, and \te{Node} components, but also their exact values.  Any
record of this type will have a {\tt Tree} etc.\ with exactly these values,
whereas the value of {\tt depth} may differ.

This idea comes from the type system for the SML module system where
these types are called translucent sums, \cite{lillibridge:translucent},
or the similar notion of singleton kinds.
%%%, \cite{harper:skind}.
A similar construct is also present in Cardelli's Quest, where it is called
manifest definitions, \cite{cardelli:quest}.

\subsection{Typing and evaluation rules}
The Cayenne typing rules are given in table~\ref{table:typing-1} and
table~\ref{table:typing-2}.

Some simplifications have been made to the typing rules for the
purpose of presentation.  In data type each constructor has exactly one
argument which must be of value type.  Furthermore, the order of
the definitions in a \te{struct}/\te{sig} does not matter in real
Cayenne, where as they do in the rules.

The stratification showed in the typing rules is also a simplification
of the one used in Cayenne.  The type of types as used in the
rules is $\hash{n}$, but in actuality it is $\hash{n,m}$.
The first subscript is derived as shown in the typing rules
and the second we get by replacing $min$ by $max$ in the {\sf Prod} rule.
The reason for two subscripts is that the first number is necessary
for getting the type erasure to be possible, and the second one is
necessary if we want the logic to be sound (if recursion
is removed).

The environment (or assumtions), $\Gamma$, in the typing rules may contain
variables with their types, as is usually the case.  But they may
also contain variables with their types as well as their {\em values}.
The reason for the values is that we sometimes need values to
enable reductions during type checking.  It is the {\sf Rec} rule
that introduces values into $\Gamma$.

The fact that Cayenne has dependent types shows up in a few places
in the typing rules.  In the {\sf App} rule the term $f\;a$ has a type
that may depend on $x$, so $x$ is replaced with the actual value in $B$.
Furthermore, in the {\sf Case} rule in each arm the type of the arm
may depend of the scrutinized variable so a substitution is performed here as well.

The translucent sums show up in the {\sf SelE} rule where a term
$e.l$ can be reduced even if only the type of $e$ is known.  This
reduction is only performed during type checking and never during normal reduction
(execution).

Because of a lack of time, we have not yet proved essential theorems
about the Cayenne type system, such as soundness and the subject
reduction theorem.  While we believe them to be true, and they have been
proved similar systems, they have not proved for a system with dependent
types and translucent sums.

The Cayenne evaluation rules, table \ref{tab:eval}, are unsurprising.
Note that because definitions in a \te{struct} are recursive
some care has to be exercised.

\subsubsection{A problem with substitution}
Substitution in Core Cayenne as described by the rules in this section
suffers from a problem: it does not really work; there are some
unavoidable name clashes.  We illustrate the problem with an example.
To make the example shorter we omit types and use a \te{let}
expression which could be translated to a record expression.
\begin{verbatim}
struct
    x = 1
    z = let y = x
        in  struct x = y
\end{verbatim}
As we can easily see the \te{z} component of this record is a record
with an \te{x} component with value \te{1}.  Let us apply the standard
reduction rule for \te{let}, namely \ieval{\keyw{let} x = e \;\keyw{in}
e'}{\subst{e'}{x}{e}}.
\begin{verbatim}
struct
    x = 1
    z = struct x = x
\end{verbatim}
This is clearly not the same value as we had before; the \te{x} has
been captured when it should not be.  Note that we cannot rename either
of the two \te{x}s since the name of the labels appear in the type;
renaming them would change the type.

This problem is annoying, but can be handled easily.  All we need to do
is to have two different names for all labels.  One name is the label
itself as it appears in the type and the other name is the name that
is bound inside the record.  The second name is not part of the type
and can clearly be $\alpha$-converted when necessary.  To avoid
cluttering the typing rules even more we will not introduce any
notation for this in the rules, instead we assume that the problem can
be handled if needed.  A similar solution is used in \cite{betarte98:thesis}.

%%%%\com{

\begin{table}

\com{
\ruleh{Empty context}
    {\mbox{}}
    {\langle\rangle\; valid}
\rsep
\ruleh{Type context}
    {\Gamma\; valid \seph \judge{\Delta}{\hastype{A}{t}}}
    {\Gamma,\hastype{x}{A}\; valid}
\rsep
\ruleh{Value context}
    {\Gamma\; valid \seph \judge{\Delta}{\hastype{A}{t}}} \seph \judge{\Delta}{\hastype{a}{A}}
    {\Gamma,\hastype{x}{A = a}\; valid}
\rsep
}
\ruleh{Star}{\mbox{}}
            {\judge{\Gamma}
                   {\hastype{\hash{n}}
                            {\hash{n+1}}}}
\rsep
\ruleh{Var}{\judge{\Gamma}{\hastype{A}{s}}}
           {\judge{\Gamma,\hastype{x}{A}}{\hastype{x}{A}}}
\rsep
\ruleh{Pi}
    {\judge{\Gamma}{\hastype{A}{s}} \seph
     \judge{\Gamma,\hastype{x}{A}}{\hastype{B}{t}}}
    {\judge{\Gamma}{\hastype{(x\dc A)\fun B}{t}}}
\rsep
\ruleh{Lam}
    {\judge{\Gamma,\hastype{x}{A}}{\hastype{b}{B}} \seph 
     \judge{\Gamma}{\hastype{(x\dc A)\fun B}{t}}}
    {\judge{\Gamma}{\hastype{\lam(x\dc A)\fun b}{(x\dc A)\fun B}}}
\rsep
\ruleh{App}{\judge{\Gamma}{\hastype{f}{(x\dc A)\fun B}} \seph \judge{\Gamma}{\hastype{a}{A}}}
           {\judge{\Gamma}{\hastype{f\;a}{\subst{B}{x}{a}}}}
\rsep
\ruleh{Data}
    {\judge{\Gamma}{\hastype{A_1}{\hash{1}}} \seph \ldots \seph
     \judge{\Gamma}{\hastype{A_n}{\hash{1}}}}
    {\judge{\Gamma}{\hastype{\;$\te{data}$\; C_1\;A_1 $\te{|}$ \ldots$\te{|}$\; C_n\;A_n }{\hash{1}}}}
\rsep
\ruleh{Con}
    {\judge{\Gamma}{\hastype{T}{\hash{1}}}}
    {\judge{\Gamma}{\hastype{C_k$\te{@}$T}{A_k \fun T}}}
where $T \equiv \;$\te{data}$\; C_1\;A_1 $\te{|}$ \ldots$\te{|}$\; C_n\;A_n$

\rsep
\ruleh{Case}
{
\begin{array}{rcl}
\judget{\Gamma}{\hastype{x}{\;$\te{data}$\; C_1\;A_1 $\te{|}$ \ldots$\te{|}$\; C_n\;A_n}} \\
\judget{\Gamma,\hastype{x_1}{A_1}}{\hastype{e_1}{\subst{A}{x}{C_1\;x_1}}} \\
&\vdots&\\
\judget{\Gamma,\hastype{x_n}{A_n}}{\hastype{e_n}{\subst{A}{x}{C_n\;x_n}}} \\
\end{array}
}
{\judge{\Gamma}
  {\hastype{$\te{case}$\;x\;$\te{of \{}$\; C_1\;x_1 \fun e_1 \sm \ldots \sm C_n\;x_n \fun e_n$\te{\}::}A$ }
           {A}
  }
}
\rsep
\ruleh{Prod}
{
\begin{array}{rcl}
\judget{\Gamma}{\hastype{A_1}{\hash{u_1}}} \\
\judget{\Gamma,\;\hastype{l_1}{A_1}}{\hastype{A_2}{\hash{u_2}}} \\
& \vdots & \\
\judget{\Gamma,\;\hastype{l_1}{A_1},\cdots,\hastype{l_{n-1}}{A_{n-1}}}{\hastype{A_n}{\hash{u_n}}} \\
& \vdots & \\
\judget{\Gamma,\;\hastype{l_1}{A_1},\cdots,\hastype{l_n}{A_n}}{\hastype{e_j}{A_j}} \\
& \vdots & \\
\end{array}
}
{\judge{\Gamma}
  {\hastype
    {$\te{sig\{}$\; l_1 \dc \gamma_1 \sm \ldots l_n \dc \gamma_n $\te{\}}$\;}
    {\hash{min\{u_i\}}}
  }
}
where each $\gamma_i$ is either $A_i$ or ``$A_i = e_j$''\\

\rsep
\ruleh{Rec}
{
\begin{array}{rcl}
\judget{\Gamma}{\hastype{A_1}{s_1}} \\
\judget{\Gamma,\Delta}{\hastype{e_1}{A_1}} \\
\judget{\Gamma,\;\hastype{l_1}{A_1}}{\hastype{A_2}{s_2}} \\
\judget{\Gamma,\Delta}{\hastype{e_2}{A_2}} \\
& \vdots & \\
\judget{\Gamma,\;\hastype{l_1}{A_1},\cdots,\hastype{l_{n-1}}{A_{n-1}}}{\hastype{A_n}{s_n}} \\
\judget{\Gamma,\Delta}{\hastype{e_n}{A_n}}
\end{array}
}
    {\judge{\Gamma}{
\begin{array}{c}
$\te{struct\{}$\; p_1\, a_1\, l_1 \dc A_1 \teq e_1 \sm \ldots  p_n\,a_n\,l_n \dc A_n \teq e_n $\te{\}}$\\
\hastype{\mbox{}}{\;$\te{sig\{}$\; \ldots l_i \dc \gamma_i \sm \ldots $\te{\}}$\;}
\end{array}
    }}
where $l_i$ is present iff $p_i =\;$\te{public},\\
$\gamma_i$ is ``$A_i = e_i$'' if $a_i =\;$\te{concrete} otherwise $A_i$\\
$\Delta \equiv \hastype{l_1}{A_1 = e_1},\cdots,\hastype{l_n}{A_n = e_n}$

\rsep
\ruleh{Sel}
    {\judge{\Gamma}{\hastype{e}{\;$\te{sig\{}$\; \ldots l_i \dc \gamma_i \sm \ldots $\te{\}}$\;}}}
    {\judge{\Gamma}{\hastype{e \dt l_i}{\subst{A_i}{\ldots,l_k}{e\dt l_k,\ldots} }}}

\caption{Core Cayenne typing rules}
\label{table:typing-1}
\end{table}

\begin{table}
\center{
\ruleh{Conv}
    {\judge{\Gamma}{\hastype{a}{A}} \seph
     \judge{\Gamma}{\hastype{B}{s}} \seph
     \judge{\Gamma}{\eq{A}{B}}}
    {\judge{\Gamma}{\hastype{a}{B}}}
\rsep
\ruleh{Weak}
    {\judge{\Gamma}{\hastype{A}{s}} \seph
     \judge{\Gamma}{\delta}}
    {\judge{\Gamma,\hastype{x}{A}}{\delta}}
\rsep
\ruleh{WeakE}
    {\judge{\Gamma}{\hastype{a}{A}} \seph
     \judge{\Gamma,\hastype{x}{A}}{\delta}}
    {\judge{\Gamma,\hastype{x}{A = a}}{\delta}}
\rsep
}
\caption{Core Cayenne typing rules, continued}
\label{table:typing-2}
\end{table}

\begin{table}
\ruleh{Congr}
    {\judge{\Gamma}{\hastype{a}{A}} \seph \judge{\Gamma}{\hastype{b}{A}} \seph \judge{\Gamma}{\eq{a}{b}}}
    {\judge{\Gamma}{\eq{C[a]}{C[b]}}}
where $C[]$ is any context

\rsep
\ruleh{Red}
    {\judge{\Gamma}{\hastype{a}{A}} \seph $\ieval{a}{b}$}
    {\judge{\Gamma}{\eq{a}{b}}}

\rsep
\ruleh{Lookup}
    {}
    {\judge{\Gamma,\hastype{x}{A = e}}{\judge{\Gamma}{\eq{x}{e}}}}

\rsep
\ruleh{SelE}
    {\judge{\Gamma}{\hastype{e}{\;$\te{sig\{}$\; \ldots l_i \dc A_i = e_i \sm \ldots $\te{\}}$\;}}}
    {\judge{\Gamma,\hastype{l_1}{A_1},\cdots,\hastype{l_n}{A_n}}{\eq{e \dt l_i}{e_i}}}

\rsep
\com{
\ruleh{Rule}
    {\judge{\Gamma}{\hastype{a}{A}}}
    {\judge{\Gamma}{\hastype{a}{A}}}
\rsep
    }

\caption{Core Cayenne equality rules}
\label{tab:eq}

\end{table}



\begin{table*}
\center{
\begin{tabular}{rcl}
\eval{\te{(}\bs$x\dc t\fun f$\te{)}$e$} {$\subst{f}{x}{e}$}
\eval{$e\dt l_k$} {$\subst{e_k}{\ldots,l_k}{e\dt l_k,\ldots}$}
\multicolumn{3}{c}{where $e \equiv$ \te{struct\{}$\ldots\;$\te{public}$\;a_k\;l_k\dc t_k\teq e_k\sm \ldots $\te{\}}} \\
\eval{\te{case}$\;C_k$\te{@}$t\;e\;$\te{of}$\;\ldots\;C_k\;x_k\fun\;e_k\sm\ldots$} { $\subst{e_k}{x_k}{e}$ }
\end{tabular}
\caption{Core Cayenne evaluation rules}
\label{tab:eval}
}
\end{table*}

\newcommand\blob{\bullet}
\newcommand\hashn{\hash{1}}
\begin{table*}
\center{
\begin{tabular}{rcl}
$($\bs$(x\dc t)\fun f)^*$ & $\rightarrow$ & \bs$x\fun f^*$, if $t\in\hashn$ \\
$($\bs$(x\dc t)\fun f)^*$ & $\rightarrow$ & $f^*$, if $t\notin\hashn$  \\
$(f\;e)^*$ & $\rightarrow$ & $f^*\;e^*$, if $e\in t$ and $t\in\hashn$ \\
$(f\;e)^*$ & $\rightarrow$ & $f^*$, if $e\in t$ and $t\notin\hashn$ \\
$((x\dc t)\fun f)^*$ & $\rightarrow$ & $\blob$ \\

\te{struct\{}$\;l_1\dc t_1\teq e_1\sm \ldots\;l_n\dc t_n\teq e_n$\te{\}}$^*$ & $\rightarrow$ 
        & \te{struct\{}$\ldots\;l_k\teq e_k^*\ldots$\te{\}}, for those $l_k$ where $t_k\in\hashn$ \\
$(e\dt l)^*$ & $\rightarrow$ & $e^*\dt l$ \\
\te{sig\{} $\;\ldots\;$ \te{\}}${}^*$ & $\rightarrow$ & $\blob$ \\

$(C$\te{@}$t)^*$ & $\rightarrow$ & $C$ \\
$($\te{case}$\;e\;$\te{of}$\;C_1\;x_1\fun \;e_1\sm \ldots\;C_n\;x_n\fun \;e_n)^*$ & $\rightarrow$ 
        & \te{case}$\;e^*\;$\te{of}$\;C_1\fun \;e_1^*\sm \ldots\;C_n\fun \;e_n^*$ \\
\te{data}$\;\ldots\;{}^*$ & $\rightarrow$ & $\blob$ \\

$\hash{n}^*$ & $\rightarrow$ & $\blob$ \\
$x^*$ & $\rightarrow$ & $x$, if $x\in t, t\in\hashn$ \\
$x^*$ & $\rightarrow$ & $\blob$, if $x\in t, t\notin\hashn$ \\

\end{tabular}
\caption{Type erasure transformation}
}
\end{table*}

\begin{table*}
\center{
\begin{tabular}{rcl}
\evalu{\te{(}\bs$x\fun f$\te{)}$e$} {$\subst{f}{x}{e}$}
\evalu{$e\dt l_k$} {$\subst{e_k}{\ldots,l_k}{e\dt l_k,\ldots}$}
\multicolumn{3}{c}{where $e \equiv$ \te{struct\{}$\ldots\;a_k\;l_k\teq e_k\sm \ldots $\te{\}}} \\
\evalu{\te{case}$\;C_k\;e\;$\te{of}$\;\ldots\;C_k\;x_k\fun\;e_k\sm\ldots$} 
  { $\subst{e_k}{x_k}{e}$ }
\end{tabular}
\caption{Core Cayenne typeless evaluation rules}
}
\end{table*}
%%%%%}

\subsection{Type checking}

Type checking of Cayenne is basically simple, just because it is
type checking rather than type deductions, like e.g.\ Haskell uses.
Type checking proceeds in a single traversal of the syntax tree.
On the way down the environment ($\Gamma$) is extended with the
types (and sometimes values) of bound identifiers.  Since Cayenne
has explicit types the type of each bound identifier is known.
On the way up the type of each subexpression can the be computed
and checked.
A complication
arises when a typing rules has more than one occurence of a type,
like $A$ in the {\sf App} rule in table \ref{table:typing-1}.
For these
cases we need to check if the two types derived from the bottom
up derivation are the same, and if they are not identical the {\sf Conv} rule
can be used to make them equal (assuming the program is type correct).
For a strongly normalizing language without translucent types the {\sf Conv}
rule is uses $=_\beta$ for $\eq{}{}$.  This relation is
easy to implement; just compute the normal forms of the two
types and compare those.  Since Cayenne is not strongly normalizing
this is not an option.  The equivalence of two arbitrary expression is
undeciable. For this reason, we can not implement anything but an
approximation of the {\sf Conv} rule and the equality rules (table \ref{tab:eq}).
This is a tricky part of the
Cayenne type checker since if the equivalence test is
implemented in a na�ve way type checking can easily loop.

%%%\subsection{Soundness}

\subsection{Undecidability in practice}
So type checking Cayenne is undecidable.  This is unfortunate, but
unavoidable for a language like Cayenne.
How bad is it in
practice to have an undecidable type checker?  This question can only
be answered by practical experiments.
The Cayenne programs we have tried
to date range from ordinary Haskell style programs, to programs
using dependent types, to proofs of mathematical propositions.
The total size of these programs are only a few thousand lines,
but so far the experience shows that it works remarkably well.  

Having undecidable type checking means
that the type checker might loop.  This is clearly not a user friendly
type checker.  So instead the implemented type checker has an upper bound on the number of
reduction steps that it may perform.  If this limit is exceeded the
type checker will
report this.  Most of the type errors from the Cayenne compiler are
similar to those that any other language would give.  Very
infrequently does the type checker report that it did not terminate within
the prescribed number of steps.  Most often, this is the result of a
type error, but sometimes the type expression is just too complicated
and the number of reduction steps must be increased (the number of
reduction steps is a compiler flag).

The type checker can thus give one of three answers: type correct (meaning
that the program will not go wrong when run), type incorrect, or ``don't know''\footnote{
On a real machine Hindley-Milner type checking has the same problem, but the
third alternative is usually spelled ``Out of memory'' instead.}.

There are other languages with undecidable type checking, e.g.,
Quest \cite{cardelli:quest} (which has a type system based on
$F_{\omega <}$) and Gofer \cite{jones:gofer-impl}, but it is usually
more difficult to make these systems loop.
%%  C++  ???

\section{Cayenne as a proof system}
Since Cayenne has unrestricted recursion, this means that every type is
inhabited by at least one element, namely $\perp$.  Thus,
proofs made in Cayenne cannot really be trusted as proofs, since any
proposition can be proved by $\perp$.  If proper checking is done, it
is often\footnote{Not always, of course, since then we would have to solve the
halting problem.}
possible to ensure that a proof is valid, but no such
checking is done at the moment.

Even if a proofs expressed in Cayenne cannot be trusted because
they pass the type checker it is still valuable to have the
encoding of predicate logic in the language.  Firstly, it allows
us to express properties about programs within the language even
if we provide no proofs at all.  It is better to have this ability
within the language than to use comments or leave out those
properties completely.
Secondly, even if a proof cannot be trusted one can argue that a
proof that has been checked, but may be $\perp$, is better than
a proof that is not checked at all.

%But even if a proof cannot be completely trusted, or even if the proof
%is omitted, it is very useful to have a type language rich enough to
%express properties that you want to hold.

\section{Implementation}
Implementing Cayenne is fairly straight forward; it is like any
other functional language.  One decision that has to be made
is what to do with types at runtime.

\subsection{Erasing types}
\label{sec:erasetypes}
Cayenne treats types like first class values.  Does that mean
that the types have to be present at run time, passed around as
arguments, stored in data structures, etc?  No, they do not.  There is no
language construct, e.g., \te{casetype}, that allows a ground value --- which is all that can be
observed in a program --- to depend on a type. Hence, types do not have to be
present at run time.  Erasing types consists of removing all arguments
and record components that have type $\hash{n}$ or are functions
computing something of type $\hash{n}$.
In \cite{cardelli:phase} it is claimed that type erasing is not possible
and that the distinction between compile-time and run-time is blurred
with dependent types.  We claim that this is not the case with
the variant of dependent types used in Cayenne.

%{\sf We have proved that reduction before and after erasing the types
%of a term gives the same result.  This proof will be in the full paper.}

\nocom{
What we need to show is that evaluating an expression with types erased
yields the same result as evaluating it with the types left in.

{\bf Definition} An expression, $e$, has {\em value type} if $e \in t$ and 
$t \in \hash{1}$

{\bf Theorem} If $e$ has value type and \imeval{e}{v} then \imevalu{e^*}{v^*}.

We first prove a useful lemma.

{\bf Lemma}
If $e$ has value type, then $e^*$ contains no $\blob$.

{\bf Proof}
We assume that the expression
to transform is of value type, and show that each invocation of the transformation
on a subexpression is also on an expression of value type.

{\bf Cases}
A $\lambda$-expression \bs$(x\dc t)\fun f$ has type $(x\dc t)\fun r$, where $r$
is the type of $f$.  According to the assumption $(x\dc t)\fun r$ has type
$\hashn$
and typing rule {\sf Pi} shows that then $r$ has type $\hashn$ as well.  Thus the
transformation of $f$ is also on an expression of value type.

For an application $f\; e$, according to the definition of ${}^*$, the transformation
is only applied to $e$ if it is of value type.  $f$ has type $(x\dc t)\fun r$ and
$f\; e$ has type $r$, if $r$ has type $\hashn$ then, again according to typing rule 
{\sf Pi}, $(x\dc t)\fun r$ has type $\hashn$, so the transformation of $f$ fulfills
the assumption.

The transformation cannot be applied to a function type since this does not
have value type.

For a record value \te{struct\{}$\; \ldots\;$\te{\}} the transformation is only
applied to subexpressions of value type according to the definition of ${}^*$.

If a record selection $e\dt l$ is of value type then the field $l$ must be of value
type.  If one field of a record type type has type $\hashn$ then the whole record
type has type $\hashn$ according to typing rule {\sf Prod} (which takes the $min$
of all the types), so the subexpression $e$ (of $e\dt l$) must have value type.

The transformation cannot be applied to a \te{sig\{}$\; \ldots\;$\te{\}} value
since it is not of value type.

The lemma is trivially true for a contructor expression.

For a \te{case} expression the transformation is applied to the scrutinized expression,
which is always of value type (typing rule {\sf Data}) and to all the right hand
sides.  The right hand sides are of value type if the whole case expressions is.

The transformation cannot be applied to a \te{data} value since it is not of value type.

The transformation cannot be applied $\hashn$ since it is not of value type.

The lemma is true for variables according to the definition of ${}^*$.

{\bf Corollary} A transformed expression of value type contains no variables that were
not of value type in the original expression.

{\bf Proof} Variables that are not of value types are translated to $\blob$,
but there are no $\blob$ in the expression, hence there can be such variables.
\medskip

{\bf Lemma} The substitution lemma states that type erasure
commutes with substittion: $(\subst{e}{x}{t})^* = \subst{e^*}{x}{t^*}$.

{\bf Proof} By structural induction over the expression syntax.
\medskip

We can now return to proving the type erasure theorem.  First we prove that
if $e$ has value type and \ieval{e}{f} then \ievalu{e^*}{f^*} or $e^* = f^*$.  The theorem then
follows simply by induction on the length of the reduction sequence.

The single step version of the theorem is proved by case analysis on the three different
(typed) reduction kinds.

{\bf Cases} If the reduction is \ieval{\te{(}$\bs$ x\dc t\fun f \te{)}e}{\subst{f}{x}{e}} then
the translation of the redex is either $\te{(}$\bs$ x\fun f^* \te{)}e^*$ in which case there
is a corresponding untyped reduction step (according to the substitution lemma).
Or the translation of the redex is $f$ (if $x$ and $e$ do not have value type).
In this case $f^* = \subst{f^*}{x}{e^*}*$ since $x$ does not occur in $f^*$
(according to the corollary).

If the reduction is a selection the selected label could either be left in the transformed
\te{struct} or it could have been erased.  But since the expression $e.l$ has value type
this means that the label has value type and it must thus be left in the \te{struct}.
There is then an exactly corresponding untyped reduction.

If the reduction is a case reduction there is an exactly corresponding untyped reduction.

QED
}

\subsection{Keeping types}
By keeping types at runtime it is possible to do computations
on types and base control decision on the dynamic type of values.
With runtime types we could have a \te{casetype} language construct.
Keeping types around at runtime have some advantages, like mostly tag-free
garbage collection, as used in TIL, \cite{morrisett:til,morrisett:phd}.

\subsection{The current implementation}
The current implementation of Cayenne is written in Haskell and
translates Cayenne to untyped LML.  The compiler consists of about
5500 lines, a third of which is the actual type checker.
The compiler parses Cayenne, does type checking and various other
checks, erases types and then translates the resulting code into LML.
The LML code is then compiled with the LML compiler, \cite{augustsson:chalmersLML},
with type checking turned off.  This works because the LML compiler does
not rely on a the fact that the program is type correct in the Hindley-Milner
type system; all the compiler assumes is that the program ``makes sense''.

A snapshot of the current
implementation can be found on the Web at\\
{\tt http://www.cs.chalmers.se/\~{}augustss/cayenne/}.



\section{Related work}
There are many logical frameworks (proof checking systems) that are based on
dependent types.  Some examples, among many, are
ALF \cite{magnussonnordstrom:alf,nordstrom:nijmegen,augustsson:alf},
CoC \cite {coquand:constructions,coquand:information},
ELF \cite{pfennig:elf,pfennig:elf91,harper:lf-jacm},
Lego \cite{pollack:thesis},
and NuPRL \cite{constable:nuprl}.
All these systems are primarily designed for making (constructive)
proofs even if many of them can also execute the resulting proofs
or extract a program from them.  Our approach is different in that we
want to make a programming language, not a proof system, but of course
there are big similarities.

There are few programming languages with dependent types.
Cardelli's Quest, \cite{cardelli:quest}, have similarities with
Cayenne, but the final version of Quest
does not have the full dependency where types
can depend on values.
Russell, \cite{boehm:russell}, has dependent types, but the notion
of type equality is ``name equality'' rather than the ``structural
equality'' of Cayenne.  Russell does not do full evaluation during
type checking so it would not be able to do, e.g., the \te{printf} 
example.  Russell also has a different notion of what a type is.


%? \cite{coquand:patternmatching,coquand:statusofpatternmatching}

\section{Future work}
There are many ways to continue the work on Cayenne and related
languages.  First, and foremost, is to gain more experience with
a language with dependent types, both to see how dependent types
can be used and to see how undecidable type checking works out.

Another interesting line of work is to make a partial evaluator for
this kind of language.  Since types and values are combined, a partial evaluator would
serve both as a type specializer 
(as used in, e.g., \cite{augustsson:overloading,jones:overloading})
and a traditional partial evaluator.

To make the record types more useful, subtyping could be added.
Subtyping in the presence of dependent types has been studied in \cite{betarte98:thesis}.

As a proof of concept the Cayenne compiler should, of course, be
rewritten in Cayenne.

%%The most immediate work, though, is to prove that the Cayenne type
%%system is sound.  

%\section{Conclusions}

\section{Acknowledgments}
A big thanks to Jessica for improving my English.  The programming logic group
at Chalmers has over the years provided me with enough background material to finally try to
make a programming language with dependent types.  A special thanks to Theirry
Coquand for fruitful discussions and examples of how to write type checkers for
dependent types.  
Thomas Johnsson, Niklas R\"ojemo and Dan Synek provided me with feedback
on this paper as did the anonymous ICFP referees.

% \tableofcontents


%\pagebreak
%%\renewcommand{\baselinestretch}{0.90}
\bibliography{/usr/local/lib/pmgrefs}
\bibliographystyle{alpha} 



%%%\com{
%\onecolumn
\pagebreak
\appendix
\section{The Eq class}
\label{app:equiv}
{
\renewcommand{\baselinestretch}{0.90}
\small
\begin{verbatim}
module example$Eq = 
#include Prelude
struct

data Absurd =

data Truth = truth

absurd :: (a :: #) |-> Absurd -> a
absurd i = case i of { }

type (<=>) a b = sig { impR :: a->b; impL :: b->a; }

concrete
Lift :: Bool -> #
Lift (False) = Absurd
Lift (True)  = Truth

concrete
LiftBin :: (a:: #) |-> (a -> a -> Bool) -> Rel a
LiftBin |a op = \(x::a) -> \(y::a) -> Lift (op x y)

type Rel a = a -> a -> #

concrete
Refl :: (a :: #) |-> Rel a -> #
Refl |a R = (x::a) -> x `R` x

concrete
Symm :: (a :: #) |-> Rel a -> #
Symm |a R = (x,y::a) -> x `R` y -> y `R` x

concrete
Trans :: (a :: #) |-> Rel a -> #
Trans |a R = (x,y,z::a) -> x `R` y -> y `R` z -> x `R` z

concrete
Equiv :: (a :: #) |-> Rel a -> #
Equiv R = sig 
  refl :: Refl R
  symm :: Symm R
  trans :: Trans R

--------
-- The Eq "class", with equivalence proof
type Eq a = sig
  (==) :: a -> a -> Bool
  equiv :: Equiv (LiftBin (==))

--------
-- Equality on Unit
Eq_Unit :: Eq Unit
Eq_Unit = struct 
  (==) (unit) (unit) = True

  equiv = struct
    refl (unit) = truth
    symm (unit) (unit) p = p
    trans (unit) (unit) (unit) p q = p

--------
-- Equality on Bool
Eq_Bool :: Eq Bool
Eq_Bool = struct
  (==) (False) (False) = True
  (==) (True)  (True)  = True
  (==) _       _       = False

  equiv = struct
    refl (False) = truth
    refl (True)  = truth
    symm (False) (False) p = p
    symm (False) (True)  p = absurd p
    symm (True)  (False) p = absurd p
    symm (True)  (True)  p = p
    trans (False) (False) (False) p q = q
    trans (False) (False) (True)  p q = absurd q
    trans (False) (True)  _     p q = absurd p
    trans (True)  (False) _       p q = absurd p
    trans (True)  (True)  (False) p q = absurd q
    trans (True)  (True)  (True)  p q = q

--------

private
liftAndL :: (x,y::Bool) -> 
            Lift (x && y) -> Pair (Lift x) (Lift y)
liftAndL (False) _       a = absurd a
liftAndL (True)  (False) a = absurd a
liftAndL (True)  (True)  t = (t, t)

private
liftAndR :: (x,y::Bool) -> 
            Pair (Lift x) (Lift y) -> Lift (x && y)
liftAndR (False) _       (a, _) = a
liftAndR (True)  (False) (_, a) = a
liftAndR (True)  (True)  (t, _) = t

private
isoEquiv :: (a :: #) |-> 
     (p, q :: Rel a) -> ((x, y :: a) -> 
     p x y <=> q x y) -> Equiv p -> Equiv q
isoEquiv p q iso eqp = struct
  refl x = (iso x x).impR (eqp.refl x)
  symm x y lp = 
    (iso y x).impR (eqp.symm x y ((iso x y).impL lp))
  trans x y z lp lq = (iso x z).impR
    (eqp.trans x y z ((iso x y).impL lp) ((iso y z).impL lq))

-- Equality on pairs.
Eq_Pair :: (a,b :: #) |-> Eq a -> Eq b -> Eq (Pair a b)
Eq_Pair eqa eqb = struct
  (==) (x, x') (y, y') = eqa.(==) x y && eqb.(==) x' y'

  private
  eq :: Pair a b -> Pair a b -> #
  eq (x, x') (y, y') = 
    Pair (LiftBin eqa.(==) x y) (LiftBin eqb.(==) x' y')

  private
  eqEq :: (x,y::Pair a b) -> eq x y <=> Lift (x == y)
  eqEq (x, x') (y, y') = struct
    impR p = liftAndR (eqa.(==) x y) (eqb.(==) x' y') p
    impL p = liftAndL (eqa.(==) x y) (eqb.(==) x' y') p

  private
  equivEq :: Equiv eq
  equivEq = struct
    refl (x, x') = (eqa.equiv.refl x, eqb.equiv.refl x')
    symm (x, x') (y, y') (pxy, pxy') = 
      (eqa.equiv.symm x y pxy, eqb.equiv.symm x' y' pxy')
    trans (x, x') (y, y') (z, z') (pxy, pxy') (pyz, pyz') = 
      (eqa.equiv.trans x y z pxy pyz, 
       eqb.equiv.trans x' y' z' pxy' pyz')

  equiv = isoEquiv eq (LiftBin (==)) eqEq equivEq
\end{verbatim}
}


\section{The tautology function}
\label{app:taut}
{
\renewcommand{\baselinestretch}{0.90}
\small
\begin{verbatim}
module example$taut =
#include Prelude
struct
data Nat = Zero | Succ Nat

concrete
TautArg :: Nat -> #
TautArg (Zero)   = Bool
TautArg (Succ m) = Bool->TautArg m

taut :: (n::Nat) -> TautArg n -> Bool
taut (Zero)   x = x
taut (Succ m) x = taut m (x True) && taut m (x False)


module example$tauttest =
#include Prelude
open example$taut use Nat, Zero, Succ, taut, TautArg in

let id :: Bool -> Bool
    id x = x

    implies :: Bool -> Bool -> Bool
    implies x y = not x || y

    equ :: Bool -> Bool -> Bool
    equ x y = implies x y || implies y x
in  do Monad_IO
       putStrLn (System$Bool.show (taut Zero True))
       putStrLn (System$Bool.show (taut (Succ Zero) id))
       putStrLn (System$Bool.show (taut (Succ (Succ Zero)) equ))
\end{verbatim}
}

\end{document}
